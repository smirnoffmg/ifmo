\section{Требования к исходным данным для проведения сертификационных испытаний.}

Требуемые исходные данные для выполнения контроля отсутствия недекларированных возможностей программного средства обработки информации:

\begin{itemize}
	\item Наименование операционной системы, под которой функционирует программное средство;
	\item Описание среды разработки программного обеспечения, используемые инструментальные средства разработки, библиотечные модули;
	\item Технологическая документация, описывающая процесс сборки (компиляции) программного обеспечения изделия;
	\item Исходные тексты программного средства обработки информации;
	\item Дистрибутив. Дистрибутив должен быть собран из представляемых исходных текстов;
	\item Исполняемые модули программного средства;
	\item Документация в соответствии с руководящим документом Гостехкомисии России защита от несанкционированного доступа к информации часть 1. программное обеспечение средств защиты информации классификация по уровню контроля отсутствия недекларированных возможностей от 4 июня 1999 г.
\end{itemize}

Требуемые исходные данные для выполнения испытаний соответствия реальных и декларируемых в документации функциональных возможностей программного средства обработки информации:

\begin{itemize}
	\item Спецификация;
	\item Технические условия или Формуляр;
	\item Ведомость эксплуатационных документов;
	\item Руководство по эксплуатации;
	\item Описание применения;
	\item Руководство оператора;
	\item Документация, описывающая условия и ограничения эксплуатации изделия, в соответствии с ведомостью эксплуатационных документов;
	\item Изделие в соответствии с комплектностью согласно Формуляру;
	\item Наименование операционной системы, под которой функционирует изделие;
	\item Наименование дополнительного программного обеспечения, необходимого для функционирования изделия.
\end{itemize}

Требуемые исходные данные для выполнения испытаний защищенности программного средства обработки информации от несанкционированного доступа к информации:

\begin{itemize}
	\item Руководство пользователя;
	\item Руководство по КСЗ (в соответствии с классом защищенности);
	\item Тестовая документация (в соответствии с классом защищенности);
	\item Конструкторская (проектная) документация (в соответствии с классом защищенности);
	\item Эксплуатационная документация в соответствии с ведомостью эксплуатационных документов;
	\item Наименование операционной системы, под которой функционирует программное средство;
	\item Наименование дополнительного программного обеспечения, необходимого для функционирования программного средства.
\end{itemize}

Требуемые исходные данные для выполнения испытаний защищенности
 АС от несанкционированного доступа к информации:
 
 \begin{itemize}
 	\item Перечень защищаемых информационных ресурсов АС и их уровень конфиденциальности;
 	\item Перечень лиц, имеющих доступ к штатным средствам АС, с указанием их уровня полномочий;
 	\item Матрица доступа или полномочий субъектов доступа по отношению к защищаемым информационным ресурсам АС;
 	\item Режим обработки данных в АС.
 \end{itemize}
\section{Теория}

Энтропией называется величина, вычисляемая по формуле:

\[
    H(X) = - \sum_{i=1}^n p_i \cdot log_a{p_i}
\]

Основание логарифма можно взять любым $a > 1$. Выбор основания равносилен выбору единицы измерения энтропии. Если за основание выбрать число 10, то говорят о «десятичных единицах» энтропии (дитах). На практике в качестве основания удобнее использовать число 2. При выполнении вычислений будем считать $a = 2$. В этом случае за единицу измерения энтропии принимается энтропия простейшей системы, которая имеет два равновозможных состояния, а сама энтропия измеряется в «двоичных единицах» или битах (bit).

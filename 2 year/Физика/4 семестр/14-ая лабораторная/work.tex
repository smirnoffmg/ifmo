\chapter{Определение термоэлектрических свойств металлического и полупроводникового образцов}

\section{Цель работы}

Определение термоэлектрических свойств металлического и полупроводникового образцов

\section{Ход работы}

\[
\alpha= 0.0360 [mV/\degree C]
\]

\[
t = t_0+\frac{\varepsilon}{\alpha}
\]



\begin{table}[H]
	\centering
	\begin{tabular}{|c|c|c|c|c|}
		\hline
		$t \degree$ эксп & $R_s$, [кОм] & $R_m$, [Ом] & $\varepsilon$, [мВ] & $t \degree$ рассчит  \\ \hline
		25 & 81 & 56.9 & 0 & 25.00\\ \hline
		30.5 & 75 & 57.3 & 0.2 & 30.56\\ \hline
		36 & 68 & 57.6 & 0.4 & 36.11\\ \hline
		41.5 & 58.9 & 58.3 & 0.6 & 41.67\\ \hline
		47 & 50.2 & 58.9 & 0.8 & 47.22\\ \hline
		52.5 & 40.6 & 60.1 & 1 & 52.78\\ \hline
		58 & 32.7 & 61.2 & 1.2 & 58.33\\ \hline
		63.5 & 26.9 & 62.2 & 1.4 & 63.89\\ \hline
		69 & 21.6 & 63.5 & 1.6 & 69.44\\ \hline
		74.5 & 17.3 & 64.8 & 1.8 & 75.00\\ \hline
		80 & 13.6 & 66.3 & 2 & 80.56\\ \hline
		85.5 & 10.6 & 68 & 2.2 & 86.11\\ \hline
		91 & 8.5 & 69.5 & 2.4 & 91.67\\ \hline
		96.5 & 7 & 70.8 & 2.6& 97.22\\ \hline
	\end{tabular}
\end{table}



Погрешность:

\[
\Delta t = \sqrt{\sum (t_\text{рассчит}-t_\text{эксп})^2}= 0.07 
\]

\begin{landscape}
	\begin{tikzpicture}
	\begin{axis}[grid=both,
	width=0.8\linewidth]
	\addplot coordinates {
		(0  , 25.00)
		(0.2, 30.56)
		(0.4, 36.11)
		(0.6, 41.67)
		(0.8, 47.22)
		(1  , 52.78)
		(1.2, 58.33)
		(1.4, 63.89)
		(1.6, 69.44)
		(1.8, 75.00)
		(2  , 80.56)
		(2.2, 86.11)
		(2.4, 91.67)
		(2.6, 97.22)
		(2.8, 102.78)
	};
\addplot coordinates {
		(0, 25)
		(0.2, 30.5)
		(0.4, 36)
		(0.6, 41.5)
		(0.8, 47)
		(1, 52.5)
		(1.2, 58)
		(1.4, 63.5)
		(1.6, 69)
		(1.8, 74.5)
		(2, 80)
		(2.2, 85.5)
		(2.4, 91)
		(2.6, 96.5)
};
	\legend {t рассчит, t эксп};
	\end{axis}
	
	\end{tikzpicture}
\end{landscape}

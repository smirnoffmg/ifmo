\section{Практика}


Создадим файл с псевдослучайным содержимым с помощью программы из листинга №1.

Сожмём файлы, используя стандартные архиваторы среды Linux:

\begin{lstlisting}
    zip random.zip random.txt
    tar -zcvf random.tar.gz random.txt
    tar -jcvf random.tar.bz2 random.txt
\end{lstlisting}

Для каждого файла найдём плотность распределения с помощью программы из листинга №2.


\begin{figure}[H]
    \caption{Распределение в файле random.txt}
    \label{fig:1}
    \centering
    \resizebox {.8\textwidth} {!} {
\begin{tikzpicture}
    \pgfplotstableread{src/d_random.csv}\data
    \pgfplotstablegetrowsof{\data}
    \pgfmathsetmacro{\N}{\pgfplotsretval}
    \begin{axis}[
        ybar,
        yticklabel={%
            \pgfmathparse{\tick*100}%
            \pgfmathprintnumber{\pgfmathresult}\,\%},
    ]
        \addplot+ [
            hist,
            y filter/.expression={y/\N},
        ] table [y index=0] {\data};
    \end{axis}
\end{tikzpicture}
}
\end{figure}

\begin{figure}[H]
    \caption{Распределение в файле random.tar.bz2}
    \label{fig:2}
    \centering
    \resizebox {.8\textwidth} {!} {

    \begin{tikzpicture}
        \pgfplotstableread{src/d_random_bz2.csv}\data
        \pgfplotstablegetrowsof{\data}
        \pgfmathsetmacro{\N}{\pgfplotsretval}
        \begin{axis}[
            ybar,
            yticklabel={%
                \pgfmathparse{\tick*100}%
                \pgfmathprintnumber{\pgfmathresult}\,\%},
        ]
    
            \addplot+ [
                hist,
                y filter/.expression={y/\N},
            ] table [y index=0] {\data};
        \end{axis}
    \end{tikzpicture}
    }
    \end{figure}

\begin{figure}[H]
    \caption{Распределение в файле random.tar.gz}
    \label{fig:3}
    \centering
    \resizebox {.8\textwidth} {!} {

    \begin{tikzpicture}
        \pgfplotstableread{src/d_random_gz.csv}\data
        \pgfplotstablegetrowsof{\data}
        \pgfmathsetmacro{\N}{\pgfplotsretval}
        \begin{axis}[
            ybar,
            yticklabel={%
                \pgfmathparse{\tick*100}%
                \pgfmathprintnumber{\pgfmathresult}\,\%},
        ]
    
            \addplot+ [
                hist,
                y filter/.expression={y/\N},
            ] table [y index=0] {\data};
        \end{axis}
    \end{tikzpicture}
    }
\end{figure}

\begin{figure}[H]
    \caption{Распределение в файле random.zip}
    \label{fig:4}
    \centering
    \resizebox {.8\textwidth} {!} {

    \begin{tikzpicture}
        \pgfplotstableread{src/d_random_zip.csv}\data
        \pgfplotstablegetrowsof{\data}
        \pgfmathsetmacro{\N}{\pgfplotsretval}
        \begin{axis}[
            ybar,
            yticklabel={%
                \pgfmathparse{\tick*100}%
                \pgfmathprintnumber{\pgfmathresult}\,\%},
        ]
    
            \addplot+ [
                hist,
                y filter/.expression={y/\N},
            ] table [y index=0] {\data};
        \end{axis}
    \end{tikzpicture}
    }
\end{figure}

Сравним получившиеся файлы, используя программу для расчёта энтропии из листинга №3.

\begin{table}[!h]
    \caption{Сравнение файлов}
    \label{tbl:1}
    \centering

    \begin{tabular}{|c|c|c|c|}
        \hline
        Название & Размер, байт & Энтропия & Степень сжатия \\ \hline
        random.txt & 1024 & 5.98258 & 100\% \\ \hline
        random.zip & 973 & 7.55285 & 95\% \\ \hline
        random.tar.bz2 & 1017 & 7.78326 & 99 \% \\ \hline
        random.tar.gz & 955 & 7.75808 & 93\% \\ \hline

    \end{tabular}
\end{table}


\section{Исходный код}


\begin{figure}[!h]
    \lstinputlisting[language=C,caption=Исходный код для генерации файла со случайным содержимым.]{src/generator.cpp}
\end{figure}
  
\begin{figure}[!h]
    \lstinputlisting[language=C,caption=Исходный код для расчёта распределения в указанном файле.]{src/density.cpp}
\end{figure}


\begin{figure}[!h]
    \lstinputlisting[language=C,caption=Исходный код для расчёта энтропии в указанном файле.]{src/entropy.cpp}
\end{figure}

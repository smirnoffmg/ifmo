\section{Ход работы}

\textbf{Задача}: резервное копирование информации со СКУД.

\begin{figure}[H]
	\centering
	\includegraphics[width=0.7\linewidth]{"img/skud"}
\end{figure}

\textbf{Резервное копирование данных} — это создание дополнительной копии файлов, которые будут сохранены в случае утери или повреждения вашего компьютера. Бэкап можно производить самостоятельно или использовать для этого специальные программы. 

«Бэкап» (от англ. backup) — «запас» (можно также перевести как «резервный» или «дублирующий»). Бэкап копия (backup copy) — создание копии файлов на дополнительном носителе информации (внешнем жёстком диске, CD/DVD-диске, флешке и т.д.). 

Способов резервного копирования существует два:
\begin{enumerate}
	\item Дифференциальное
	\item Полное
\end{enumerate}

При полном копировании копируются все важные файлы, а при дифференциальном - только новые за период.

В представленной схеме имеет смысл производить дифференциальное резервное копирование каждый день, а полное - раз в отчётный период, например, раз в неделю.

В случае аварийного выхода из строя основного сервера следует произвести "горячее" переподключение резервного сервера.

Резервное копирование следует производить с резервного сервера СКУД.



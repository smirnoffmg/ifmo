\section*{Введение}
\addcontentsline{toc}{section}{Введение}%

\textbf{Целью} данной работы будет сравнение двух распространённых сканеров уязвимостей;

Сканеры уязвимостей — это программные или аппаратные средства, служащие для осуществления диагностики и мониторинга сетевых компьютеров, позволяющее сканировать сети, компьютеры и приложения на предмет обнаружения возможных проблем в системе безопасности, оценивать и устранять уязвимости.

Поскольку лучший способ защиты - представить, что ты нападаешь, то использование сканеров уязвимостей для проверки защищённости своей сети кажется очень хорошим решением.

Сканеры уязвимостей, как и практически любое программное обеспечение, бывают проприетарные и с открытым исходным кодом. В данной работе будут рассмотрены сканеры с коммерческой поддержкой и закрытым исходным кодом. \textbf{Задачей} будет проверить основную функций сканеров - умение находить уязвимости.
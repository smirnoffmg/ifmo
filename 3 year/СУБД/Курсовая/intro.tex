\section*{Введение}
\addcontentsline{toc}{section}{Введение}%

Развитие средств вычислительной техники обеспечило условия для создания и широкого распространения систем обработки данных разнообразного назначения. Разрабатываются информационные системы для обслуживания различных систем деятельности, системы управления хозяйственными и техническими объектами, модельные комплексы для научных исследований, системы автоматизации проектирования и производства, всевозможные тренажеры и обучающие системы. 

Одной из важных предпосылок создания таких систем стала возможность оснащения их памятью для накопления, хранения и систематизация больших объемов данных. Другой существенной предпосылкой нужно признать разработку подходов, а также создание программных и технических средств конструирования систем, предназначенных для коллективного пользования. В этой связи потребовалось разработать специальные методы и механизмы управления  такого рода совместно используемыми ресурсами данных, которые стали называться базами данных. Исследования и разработки, связанные с проектированием, созданием и эксплуатации баз данных, а также необходимых для этих целей языковых и программных инструментальных средств  привели к появлению самостоятельной ветви информатики, получившей название системы управления данными.

Такие программные комплексы выполняют довольно сложный набор функций, связанный с централизованным управлениям  данными в базе данных и всей совокупности ее пользователей. По существу, система управления базами данных служит посредником между пользователями и базой данных.

\textbf{Целью} данной работы будет проектирование и разработка БД и приложения для управления работой цветочного магазина.


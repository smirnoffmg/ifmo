\section{Шифр простой замены}

Дан шифр-текст:

34 28 68 91 13 83 10 65 27 68 49 10 26 65 27 68 75 26 39 78 53 75 83 53
18 26 36 62 91. 26 10 74 53 13 49 10 83 10 65 53 53 36 68 72 28 10 28 13 18 86
10 27 53 75 39 83 68 57 26 18 10 91 53 57 36 53 65 28 68 91 10, 83 68 75
27 13 34 13 24 13 18 53 36 74 53 36 10 74 10 36 57 36 13, 83 68 74 10
91 10 91 10 36 13 68 26 74 18 62 34 10 27 10 36 10 75 26 13 86 39 68 74 36
10. 83 18 10 34 28 10, 26 57 26 50 62 27 68 83 68 65 57 86 13. 26 57 26 49 10
83 10 65 53 34 19 13 27 53 75 39 53 34 75 13 75 68 50 68 15 83 18 68 83 53
26 10 27 53. 49 10 83 10 65 53 10 27 74 68 72 68 27 44, 83 68 28 72 68 18 13 34 80
13 72 68 91 10 75 27 10, 83 68 26 10, 75 26 10 18 68 15 68 28 13 86 28 62 53
13 96 13 27 13 74 10 18 75 26 34 - 91 13 36 26 68 27 10 53, 74 10 86 13 26 75
44, 34 10 27 13 18 39 44 36 74 53. 34 83 18 53 65 68 86 13 15 26 13 91 36 68 26
53 96 10, 53 18 44 28 68 91 23 26 68 26 28 78 75 75 10 36 28 13 18 - 34
26 44 36 57 27 72 68 27 68 34 57 34 34 68 18 68 26, 23 26 10 74 53 15 72 18
53 47 – 75 26 13 18 34 44 26 36 53 74, 86 28 57 96 53 15, 74 68 72 28 10 18 10
36 13 36 68 13 86 53 34 68 26 36 68 13 53 75 83 57 75 26 53 26 28 57 65.

Составим таблицу встречаемости.

\begin{table}[H]
	\centering
	\begin{tabular}{|c|c|c|c|}
		\hline
	Число & Встречаемость & Число & Встречаемость \\ \hline
		10 & 39 & 65 & 9  \\ \hline
		68 & 36 & 86 & 8  \\ \hline
		26 & 29 & 72 & 7  \\ \hline
		53 & 29 & 44 & 6  \\ \hline
		13 & 28 & 39 & 5  \\ \hline
		36 & 19 & 15 & 5  \\ \hline
		18 & 18 & 49 & 4  \\ \hline
		75 & 17 & 62 & 4  \\ \hline
		34 & 16 & 96 & 3  \\ \hline
		83 & 16 & 78 & 2  \\ \hline
		27 & 16 & 50 & 2  \\ \hline
		28 & 14 & 23 & 2  \\ \hline
		74 & 13 & 24 & 1  \\ \hline
		57 & 11 & 19 & 1  \\ \hline
		91 & 10 & 80 & 1  \\ \hline
		47 & 1 & & \\ \hline
	\end{tabular}
\end{table}

Сопоставив частоту встречаемости чисел с частой встречаемости букв русского алфавита, получим следующую таблицу:

\begin{table}[H]
	\centering
	\begin{tabular}{|c|c|c|c|}
				\hline
				Код & Символ & Код & Символ \\ \hline
				34 & в & 18 & р \\ \hline
				28 & д & 36 & н \\ \hline
				68 & о & 62 & ы \\ \hline
				91 & м & 74 & к \\ \hline
				13 & е & 72 & г \\ \hline
				83 & п & 86 & ж \\ \hline
				10 & а & 57 & у \\ \hline
				65 & х & 24 & ч \\ \hline
				27 & л & 50 & б \\ \hline
				49 & з & 44 & я \\ \hline
				26 & т & 80 & ш \\ \hline
				75 & с & 19 & ъ \\ \hline
				39 & ь & 15 & й \\ \hline
				78 & ю & 96 & щ \\ \hline
				53 & и & 23 & э \\ \hline
				47 & ф & & \\ \hline
	\end{tabular}
\end{table}

Ответ: <<В доме пахло затхлостью и спиртным. Такие запахи иногда держались по утрам и у них дома, после вечеринки накануне, пока
мама не открывала настежь окна. Правда, тут было похуже. Тут запах и въелись и все собой пропитали. Запахи алкоголя, подгоревш
его масла, пота, старой одежды и еще лекарств - ментола и, кажется, валерьянки. В прихожей темнотища, и рядом этот дюссандер - втянул голову в ворот, этакий гриф – стервятник, ждущий, когда раненое животное испустит дух.>>




\section{Введение}

\subsection{Цель работы}

Цель работы - исследование зависимости сопротивления резисторов от температуры и определение ТКС различных типов резисторов.

\subsection{Краткие теоретические сведения}

В процессе эксплуатации сопротивление резисторов может значительно изменяться за счет воздействия различных возмущающих факторов. Наиболее существенное влияние оказывает изменение температуры резистора, приводящее к изменению удельного сопротивления материала токопроводящего слоя и его геометрических размеров.

Для количественной оценки температурной стабильности сопротивления резисторов пользуются величиной температурного коэффициента (ТКС), который определяется как относительное изменение сопротивления резистора при изменении температуры окружающей среды на $1^\circ C$.

\[
\alpha_R = \frac{1}{R} \frac{dR}{dt},
\]

поскольку:

\[
R = \rho \frac{l}{S}
\]


\[
\alpha_R = \frac{1}{R} \frac{dR}{dt} = \frac{1}{\rho} \frac{d \rho}{dt} + \frac{1}{l} \frac{dl}{dt} - \frac{1}{S} \frac{dS}{dt},
\]

т.е величина ТКС зависит от температурного коэффициента удельного сопротивления токопроводящего материала ($\alpha_\rho$) и от коэффициента линейного расширения ($\alpha_l$) этого материала.

Однако почти всегда $\alpha_\rho >> \alpha_l$ , по этому характер температурной зависимости сопротивления резисторов определяется в основном изменением удельного сопротивления материала токопроводящего слоя.

Температурный коэффициент может иметь различную величину и знак для одного и того же токопроводящего материала (например, композиции), поэтому практически представляет интерес не величина температурного коэффициента, а относительное изменение сопротивления резистора в определенном, достаточно широком интервале температур.

В связи с этим обычно определяется среднее значение ТКС для заданного интервала температур:

\[
\alpha_{ср}=\frac{R_1-R_2}{R_1 \cdot (t_1-t_2)},
\]

где $R_1$ , $R_2$ - величина резистора при температурах $t_1$ , $t_2$ .

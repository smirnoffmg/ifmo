\section{Инспекционный контроль.}

Проводится в случае внесения в сертифицированное изделие незначительных изменений и доработок, а также для продления срока действия сертификата соответствия.

Испытательная лаборатория должна убедиться в неизменности сертифицируемых параметров изделия.

Заявка и решение на инспекционный контроль не требуются в системе сертификации ФСТЭК России.

Орган по сертификации не привлекается

Критерии:
\begin{itemize}
	\item Незначительные изменения или доработки изделия
	\item Объем изменений в исходных текстах ПО не превышает 15\%
	\item Неизменны версия изделия и требования, указанные в сертификате соответствия
\end{itemize}

 Порядок проведения инспекционного контроля
 \begin{itemize}
 	\item Заявитель направляет в ИЛ извещение об изменениях
 	\item Испытательная лаборатория проводит анализ влияния изменений на сертифицированный функционал, проводит все испытания (НДВ, функциональные проверки) в части измененных компонент изделия
 	\item Материалы по результатам инспекционного контроля вместе с извещением об изменениях направляются в Федеральный орган по сертификации
 	\item Извещение об изменениях согласуется с Федеральным органом по сертификации
 \end{itemize}
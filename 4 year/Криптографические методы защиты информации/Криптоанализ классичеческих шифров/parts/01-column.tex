\section{Шифр столбцовой перестановки}

Дан шифр-текст: \textbf{НЗМАЕЕАА\_Г\_НОТВОССОТЬЯАЛС}.

Текст содержит 25 символов, что позволяет записать его в квадратную матрицу 5х5. Известно, что шифрование производилось по столбцам, следовательно, расшифрование следует проводить, меняя порядок столбцов

\begin{table}[H]
	\centering
	\begin{tabular}{|c|c|c|c|c|}
		\hline
		Н & З & М & А & Е \\ \hline
		Е & А & А & \_  & Г \\ \hline
		\_ & Н & О & Т & В \\ \hline
		О & С & С & О & Т \\ \hline
		Ь & Я & А & Л & С \\ \hline
	\end{tabular}
	\caption{Шифр-текст в виде матрицы}
\end{table}

Судя по таблице сочетаемости, наиболее вероятным будет появление букв в первой строке в таком порядке: \textbf{З А М Е Н}. Перепишем матрицу с учетом этого порядка.

\begin{table}[H]
	\centering
\begin{tabular}{|c|c|c|c|c|}
	\hline
	З & А  & М & Е & Н  \\ \hline
	А & \_ & А & Г & Е  \\ \hline
	Н & Т  & О & В & \_ \\ \hline
	С & О  & С & Т & О  \\ \hline
	Я & Л  & А & С & Ь  \\ \hline
\end{tabular}
	\caption{Матрица после изменения порядка столбцов}
\end{table}

Получаем осмысленный текст: \textbf{ЗАМЕНА\_АГЕНТОВ\_СОСТОЯЛАСЬ}
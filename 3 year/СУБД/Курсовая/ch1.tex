\section{Теоретические основы разработки}

\subsection{Описание предметной области}

ООО <<Цветочная мастерская>> занимается составлением и продажей букетов и сопутствующих товаров в розницу.

Для ускорения процесса обработки заказов было принято решение о создании данной информационной системе с последующем размещением в сети Интернет.

Оператора информационной системы должен иметь возможность управлять объектами продажи (букетами и сопутствующими товарам) и обрабатывать заказы от клиентов.

Клиент должен иметь возможность ознакомиться с номенклатурой товаров и заказать необходимые ему позиции без контакта с оператором.

\subsection{Анализ методов решения}

Поскольку в план автоматизации ООО <<Цветочная лавка>> не входит задачи по бухгалтерскому и управленческому учёту, то нет смысла использовать большие системы автоматизации (ERP-системы) или управления отношениями с клиентами (CRM-системы). При последующем росте компании и оборота к этому вопросу можно будет вернуться. 

На данном этапе подходящим решением оказывается использование интернет-магазина.

\subsection{Обзор средств программирования}

На момент написания данной работы топ-5 языков программирования для веб-разработки выглядит следующим образом:

\begin{itemize}
	\item PHP
	\item Python
	\item JavaScript
	\item Java
	\item Ruby
\end{itemize}

Для каждого языка существует большое количество готовых решений в области разработки интернет-магазинов. Основным являются:

\begin{itemize}
	\item Magento
	\item 1C: Bitrix
	\item Satchless
	\item Django-Oscar
	\item Node.js E-shop
	\item Prime Fusion
	\item Reaction Commerce
	\item TraderIO
\end{itemize}

\subsection{Обзор систем управления базами данных}

Несмотря на то, что все системы управления базами данных выполняют одну и ту же основную задачу (т.е дают возможность пользователям создавать, редактировать и получать доступ к информации, хранящейся в базах данных), сам процесс выполнения этой задачи варьируется в широких пределах. Кроме того, функции и возможности каждой СУБД могут существенно отличаться. Различные СУБД документированы по-разному: более или менее тщательно. По-разному предоставляется и техническая поддержка.

При сравнении различных популярных баз данных, следует учитывать, удобна ли для пользователя и масштабируема ли данная конкретная СУБД, а также убедиться, что она будет хорошо интегрироваться с другими продуктами, которые уже используются. Кроме того, во время выбора следует принять во внимание стоимость системы и поддержки, предоставляемой разработчиком.

Если речь идёт о выборе СУБД для предприятия, то следует принять во внимание возможность СУБД «расти» вместе с развитием организации. Малому бизнесу могут потребоваться только базовые функции и возможности, а также небольшое количество информации, размещаемой в БД. Но требования могут существенно расти с течением времени, а переход на другую СУБД может стать проблемой.

\subsubsection{Oracle 12c}

Неудивительно, что корпорация Oracle предлагает одноимённый продукт, с которого обычно начинается рассмотрение вариантов популярных СУБД. Первая версия Oracle была создана в конце 70-х годов, имея на данный момент блестящую репутацию. Кроме того, существует несколько версий этого продукта для удовлетворения потребностей конкретной организации.

Актуальная версия Oracle на момент написания настоящей статьи - 12с - предназначена для облачных сред и может быть размещена на одном или нескольких серверах, это позволяет управлять базами данных, которые содержат миллиарды записей. Некоторые из функций новейшей версии Oracle включают в себя GridFramework и использования как физических, так и логических структур.

Это означает, что физическое управление данными не влияет на доступ к логическим структурам. Кроме того, безопасность в этой версии доведена до высочайшего уровня, потому что каждая транзакция изолирована от других.

Достоинства:

\begin{itemize}
\item Самые свежие инновации и впечатляющий функционал уже внедрены в этом продукте, поскольку компания Oracle стремится держать планку даже на фоне других разработчиков СУБД.
\item СУБД от Oracle является крайне надёжной, фактически это эталон надёжности среди подобных систем.
\end{itemize}

Недостатки:
\begin{itemize}
	\item Стоимость Oracle может оказаться непомерно высокой, особенно для небольших организаций.
	\item Система может потребовать значительных ресурсов уже сразу после установки, поэтому возможно потребуется модернизировать оборудование для внедрения Oracle.
\end{itemize}

Идеально подходит для крупных организаций, которые работают с огромными базами данных и разнообразными функциями.

\subsubsection{MySQL}

MySQL - одна из самых популярных баз данных для веб-приложений. Фактически, является стандартом de-facto для веб-серверов, которые работают под управлением операционной системы Linux. MySQL - это бесплатный пакет программ, однако новые версии выходят постоянно, расширяя функционал и улучшая безопасность. Существуют специальные платные версии, предназначенные для коммерческого использования. В бесплатной версии наибольший упор делается на скорость и надежность, а не на полноту функционала, который может стать и достоинством и недостатком - в зависимости от области внедрения.

Разработку и поддержку MySQL осуществляет корпорация Oracle, получившая права на торговую марку вместе с поглощённой Sun Microsystems, которая ранее приобрела шведскую компанию MySQL AB. Продукт распространяется как под GNU General Public License, так и под собственной коммерческой лицензией. Помимо этого, разработчики создают функциональность по заказу лицензионных пользователей. Именно благодаря такому заказу почти в самых ранних версиях появился механизм репликации.

Эта СУБД позволяет выбирать различные движки для системы хранения, которые позволяют менять функционал инструмента и выполнять обработку данных, хранящихся в различных типах таблиц. Гибкость СУБД MySQL обеспечивается поддержкой большого количества типов таблиц: пользователи могут выбрать как таблицы типа MyISAM, поддерживающие полнотекстовый поиск, так и таблицы InnoDB, поддерживающие транзакции на уровне отдельных записей. Более того, СУБД MySQL поставляется со специальным типом таблиц EXAMPLE, демонстрирующим принципы создания новых типов таблиц. Благодаря открытой архитектуре и GPL-лицензированию, в СУБД MySQL постоянно появляются новые типы таблиц. Она также имеет простой в использовании интерфейс, и пакетные команды, которые позволяют удобно обрабатывать огромные объемы данных. Система невероятно надежна и не стремится подчинить себе все доступные аппаратные ресурсы.

Достоинства:

\begin{itemize}
	\item Распространяется бесплатно
	\item 	Прекрасно документирована
	\item 	Предлагает много функций, даже в бесплатной версии
	\item 	Пакет MySQL включен в стандартные репозитории наиболее распространённых дистрибутивов операционной системы Linux, что позволяет устанавливать её элементарно
\end{itemize}

Недостатки:

\begin{itemize}
	\item Придётся потратить много времени и усилий, чтобы заставить MySQL выполнять несложные задачи, хотя другие системы делают это автоматически, например: создавать инкрементные резервные копии
	\item Для бесплатной версии доступна только платная поддержка.
\end{itemize}

Идеально подходит для организаций, которым требуется надежный, но бесплатный, инструмент управления базами данных.

\subsubsection{Microsoft SQL Server}

Ещё одной из популярных СУБД является программный продукт Microsoft SQL-сервер. Это система управления базами данных, движок которой работает на облачных серверах, а также локальных серверах, причем можно комбинировать типы применяемых серверов одновременно. Вскоре после выпуска Microsoft SQL сервер 2016, Microsoft адаптировала продукт для операционной системы Linux, а на Windows-платформе он работал изначально.

Одной из уникальных особенностей версии 2016 года является temporal data support (временная поддержка данных), которая позволяет отслеживать изменения данных с течением времени. Последняя версия Microsoft SQL-сервер поддерживает dynamic data masking (динамическую маскировку данных), которая гарантирует, что только авторизованные пользователи будут видеть конфиденциальные данные.

Достоинства:

\begin{itemize}
\item Продукт очень прост в использовании
\item Текущая версия работает быстро и стабильно
\item Движок предоставляет возможность регулировать и отслеживать уровни производительности, которые помогают снизить использование ресурсов.
\item Вы сможете получить доступ к визуализации на мобильных устройствах.
\item Он очень хорошо взаимодействует с другими продуктами Microsoft.
\end{itemize}

Недостатки:

\begin{itemize}
\item Цена для юридических лиц оказывается неприемлемой для большей части организаций.
\item Даже при тщательной настройке производительности корпорация SQL Server способен занять все доступные ресурсы.
\item Сообщается о проблемах с использованием службы интеграции для импорта файлов.
\end{itemize}

Идеально подходит для: крупных организаций, которые уже используют ряд продуктов Microsoft.

\subsubsection{PostgreSQL}

PostgreSQL является одним из нескольких бесплатных популярных вариантов СУБД, часто используется для ведения баз данных веб-сайтов. Это была одна из первых разработанных систем управления базами данных, поэтому в настоящее время она хорошо развита, и позволяет пользователям управлять как структурированными, так и неструктурированными данными. Может быть использован на большинстве основных платформ, включая Linux. Прекрасно справляется с задачами импорта информации из других типов баз данных с помощью собственного инструментария.

Движок БД может быть размещен в ряде сред, в том числе виртуальных, физических и облачных. Самая свежая версия, PostgreSQL 9.5, предлагает обработку больших объемов данных и увеличение числа одновременно работающих пользователей. Безопасность была улучшена благодаря поддержке DBMS\_SESSION.

Достоинства:
\begin{itemize}
	\item Является масштабируемым и способен обрабатывать терабайты данных.
	\item Поддерживает формат json.
	\item Существует множество предопределенных функций.
	\item Доступен ряд интерфейсов.
\end{itemize}

Недостатки:

\begin{itemize}
	\item Документация туманна, поэтому, возможно, ответы на некоторые вопросы придется искать в интернете.
	\item Конфигурация может смутить неподготовленного пользователя.
	\item Скорость работы может падать во время проведения пакетных операций или выполнения запросов чтения.
\end{itemize}

Идеально подходит для организаций с ограниченным бюджетом, но квалифицированными специалистами, когда требуется возможность выбрать свой интерфейс и использовать json.

\subsection{Обоснование выбора системы управления базами данных}

Поскольку в разработке мы ограничены бюджетом и не можем позволить себе платную поддержку, однако используем специалистов с хорошим опытом, то будет выбрана PostgreSQL.
%!TEX TS-program = xelatex


\documentclass[a4paper,10pt]{article}

%%% Работа с русским языком
\usepackage[english,russian]{babel}   %% загружает пакет многоязыковой вёрстки
\usepackage{fontspec}      %% подготавливает загрузку шрифтов Open Type, True Type и др.
\defaultfontfeatures{Ligatures={TeX},Renderer=Basic}  %% свойства шрифтов по умолчанию
\setmainfont[Ligatures={TeX,Historic}]{Times New Roman} %% задаёт основной шрифт документа
\setsansfont{Comic Sans MS}                    %% задаёт шрифт без засечек
\setmonofont{Courier New}
\usepackage{indentfirst}
\frenchspacing

\renewcommand{\epsilon}{\ensuremath{\varepsilon}}
\renewcommand{\phi}{\ensuremath{\varphi}}
\renewcommand{\kappa}{\ensuremath{\varkappa}}
\renewcommand{\le}{\ensuremath{\leqslant}}
\renewcommand{\leq}{\ensuremath{\leqslant}}
\renewcommand{\ge}{\ensuremath{\geqslant}}
\renewcommand{\geq}{\ensuremath{\geqslant}}
\renewcommand{\emptyset}{\varnothing}

%%% Дополнительная работа с математикой
\usepackage{amsmath,amsfonts,amssymb,amsthm,mathtools} % AMS
\usepackage{icomma} % "Умная" запятая: $0,2$ --- число, $0, 2$ --- перечисление

%% Номера формул
%\mathtoolsset{showonlyrefs=true} % Показывать номера только у тех формул, на которые есть \eqref{} в тексте.
%\usepackage{leqno} % Нумерация формул слева

%% Свои команды
\DeclareMathOperator{\sgn}{\mathop{sgn}}

%% Перенос знаков в формулах (по Львовскому)
\newcommand*{\hm}[1]{#1\nobreak\discretionary{}
{\hbox{$\mathsurround=0pt #1$}}{}}

%%% Работа с картинками
\usepackage{graphicx}  % Для вставки рисунков
\graphicspath{{images/}{images2/}}  % папки с картинками
\setlength\fboxsep{3pt} % Отступ рамки \fbox{} от рисунка
\setlength\fboxrule{1pt} % Толщина линий рамки \fbox{}
\usepackage{wrapfig} % Обтекание рисунков текстом

%%% Работа с таблицами
\usepackage{array,tabularx,tabulary,booktabs} % Дополнительная работа с таблицами
\usepackage{longtable}  % Длинные таблицы
\usepackage{multirow} % Слияние строк в таблице

%%% Теоремы
\theoremstyle{plain} % Это стиль по умолчанию, его можно не переопределять.
\newtheorem{theorem}{Теорема}[section]
\newtheorem{proposition}[theorem]{Утверждение}
 
\theoremstyle{definition} % "Определение"
\newtheorem{corollary}{Следствие}[theorem]
\newtheorem{problem}{Задача}[section]
 
\theoremstyle{remark} % "Примечание"
\newtheorem*{nonum}{Решение}

%%% Программирование
\usepackage{etoolbox} % логические операторы


%%% Страница
\usepackage{extsizes} % Возможность сделать 14-й шрифт
\usepackage{geometry} % Простой способ задавать поля
	\geometry{top=15mm}
	\geometry{bottom=15mm}
	\geometry{left=15mm}
	\geometry{right=15mm}
 %
%\usepackage{fancyhdr} % Колонтитулы
% 	\pagestyle{fancy}
 	%\renewcommand{\headrulewidth}{0pt}  % Толщина линейки, отчеркивающей верхний колонтитул
% 	\lfoot{Нижний левый}
% 	\rfoot{Нижний правый}
% 	\rhead{Верхний правый}
% 	\chead{Верхний в центре}
% 	\lhead{Верхний левый}
%	\cfoot{Нижний в центре} % По умолчанию здесь номер страницы

\usepackage{setspace} % Интерлиньяж
%\onehalfspacing % Интерлиньяж 1.5
%\doublespacing % Интерлиньяж 2
\singlespacing % Интерлиньяж 1

\usepackage{lastpage} % Узнать, сколько всего страниц в документе.

\usepackage{soul} % Модификаторы начертания

\usepackage{hyperref}
\usepackage[usenames,dvipsnames,svgnames,table,rgb]{xcolor}
\hypersetup{				% Гиперссылки
    unicode=true,           % русские буквы в раздела PDF
    pdftitle={Заголовок},   % Заголовок
    pdfauthor={Автор},      % Автор
    pdfsubject={Тема},      % Тема
    pdfcreator={Создатель}, % Создатель
    pdfproducer={Производитель}, % Производитель
    pdfkeywords={keyword1} {key2} {key3}, % Ключевые слова
    colorlinks=true,       	% false: ссылки в рамках; true: цветные ссылки
    linkcolor=red,          % внутренние ссылки
    citecolor=black,        % на библиографию
    filecolor=magenta,      % на файлы
    urlcolor=cyan           % на URL
}

\usepackage{csquotes} % Еще инструменты для ссылок

%\usepackage[style=authoryear,maxcitenames=2,backend=biber,sorting=nty]{biblatex}

\usepackage{multicol} % Несколько колонок

\usepackage{tikz} % Работа с графикой
\usepackage{pgfplots}
\usepackage{pgfplotstable}


\begin{document} % конец преамбулы, начало документа
\begin{enumerate}

	\item Какие типы документов входят в систему <<Организационно-распорядительная документация>> и их назначение?
	
	Можно выделить следующие основные группы документов:
	1. организационные (уставы, положения, штатное расписание, должностные инструкции, правила внутреннего трудового распорядка);
	2. распорядительные (приказы по основной деятельности, распоряжения, решения);
	3. справочно-информационные (акты, письма, факсы, докладные записки, справки, телефонограммы);
	4. документы по личному составу предприятия (приказы по личному составу, трудовые контракты, личные дела, личные карточки по форме Т-2, лицевые счета по зарплате, трудовые книжки);
	5. коммерческие документы (контракты, договоры);
	

	\item ГОСТ Р 7.0.97-2016 является документом рекомендательного характера?
	
	Да. ГОСТ Р 7.0.97-2016 нормативно-правовым документом не является и обязательному применению не подлежит.

	\item Какое количество реквизитов определено в ГОСТ Р 7.0.97-2016?
	
	30

	\item Какие реквизиты должны быть обязательно указаны во всех внутренних
документах?

Наименование организации (должностного лица) - автора документа, название вида документа, заголовок к тексту, дата и индекс документа, текст, подпись.

	\item В каких реквизитах инициалы указывают перед фамилией, а в каких после (в соответствии с ГОСТ Р 7.0.97-2016)?
	
	?
	
	\item Какие части включает текст приказа? Что представляют собой эти части?
	
	Текст приказа состоит из двух частей: констатирующей (преамбулы) и распорядительной.
	
	Констатирующая часть отделяется от распорядительной словом "ПРИКАЗЫВАЮ", которое печатается с новой строки от поля прописными буквами или строчными в разрядку. Констатирующая часть может отсутствовать, если предписываемые действия не нуждаются в разъяснении или обосновании.
	
	Распорядительная часть должна содержать перечисление предписываемых действий с указанием исполнителя каждого действия и сроков исполнения. При этом распорядительная часть делится на пункты, если исполнение приказа предполагает несколько исполнителей и выполнение различных по характеру действий. Пункты, которые включают управленческие действия, носящие распорядительный характер, начинаются с глагола в неопределенной форме. Например: "1. Создать рабочую группу в составе...". 
	

	\item Какие требования предъявляют к формату бланков организационно-
распорядительных документов?

?

	\item Какие реквизиты обязательно указываются во всех внешних документах?
	
	Для внешних документов (входящих и исходящих) обязательными являются реквизиты: наименование организации, справочные данные об организации, наименование вида документа (кроме писем), дата документа, регистрационный номер документа, адресат, текст документа, подпись, оттиск печати

	\item Информационно-справочные документы – виды и функции?
	
	Информационно-справочные документы бывают трёх видов:
	
	1) Справочная документация (имеет индивидуальный характер, выдается по разовому запросу, касающегося одного работника, в зависимости от адресата может быть внутренней или внешней).
	
	2) Отчетно-справочная (характеризуется четкой периодичностью (годовой, квартальной и тому подобное) представления её адресатам и наличием типовых форм, сводных таблиц, итогов, в зависимости от адресата может быть внутренней или внешней (например, отчетная документация, представляемая в государственные органы статистики).
	
	3) Справочно-аналитическая (выдается по разовым или периодическим запросам с различных уровней управления, имеет сводный, обобщающий характер, разнообразие запрашиваемых показателей (пол, возраст, образование, наличие правительственных наград и так далее), в зависимости от адресата может быть внутренней или внешней).
	
	К таким документам относятся: справка, докладная записка, объяснительная записка, акт, официальное письмо, телеграмма и др.

	\item Правила оформления реквизита №11 «Регистрационный номер документа».
	
	Регистрационный номер документа (реквизит 12) состоит из его порядкового номера, который можно дополнять по усмотрению организации индексом дела по номенклатуре дел, информацией о корреспонденте, исполнителях и др.2 При этом порядковые номера для распорядительных и информационно-справочных документов присваиваются в пределах календарного года отдельно для каждого вида документа.
	
	Порядок регистрации документов и структура регистрационных номеров устанавливаются в инструкции по делопроизводству организации и распорядительных документах организации.
	Регистрационный номер документа, составленного совместно двумя и более организациями, состоит из регистрационных номеров документа каждой из этих организаций, проставляемых через косую черту в порядке указания авторов в документе.
	
	Например: распоряжение Администрации Президента Российской Федерации и Аппарата Правительства Российской Федерации от 21 декабря 2007 г. № 1576/954.
	
	Регистрационный номер документа относится к реквизитам, для которых на бланке проставляются отметки
	

	\item Правила оформления реквизита №12 <<Ссылка на регистрационный номер и дату поступившего документа>>.
	
	Ссылка на регистрационный номер и дату документа (реквизит 13) включает в себя регистрационный номер и дату документа, на который должен быть дан ответ.
	
	Реквизит "Ссылка на регистрационный номер и дату документа" оформляется преимущественно в письмах-ответах. Сведения в реквизит переносятся с поступившего документа и соответствуют регистрационному номеру и дате поступившего документа.
	
	Наличие этого реквизита исключает необходимость упоминания номера и даты поступившего документа в тексте письма, что освобождает текст от чисто вспомогательной, справочной информации. 
	
	На бланках писем проставляются отметки для этого реквизита, оформленные следующим образом: На № \rule{3em}{.1pt} от\rule{3em}{.1pt}.
	

	\item Чем отличается официальное письмо от всех других документов?
	
	Официальное письмо является единственным документом, на котором не указывается наименование. Остальные документы, например приказ, акт, идут со своим названием.

	\item Требования к оформлению бланка документов?
	
	Устанавливают два стандартных формата бланков документов - А4 (210 x 297 мм) и А5 (148 x 210 мм).
	Каждый лист документа, оформленный как на бланке, так и без него, должен иметь поля не менее:
	20 мм - левое;
	10 мм - правое;
	20 мм - верхнее;
	20 мм - нижнее.
	

	\item Правила оформления реквизита №10 «Дата документа».
	
	Дата документа проставляется должностным лицом, подписывающим или утверждающим документ, или службой документационного обеспечения управления (ДОУ) при регистрации документа, или непосредственно составителем при подготовке документа (докладная, служебная записка, заявление и др.). 
	Если авторами документа выступают несколько организаций (две или более), то датой документа считается дата наиболее позднего (последнего) подписания. 
	Дату документа оформляют арабскими цифрами в последовательности: день месяца, месяц, год. День месяца и месяц оформляют двумя парами арабских цифр, разделенными точкой; год – четырьмя арабскими цифрами.
	Например, дату 5 июня 2009 г. следует оформлять 05.06.2009.
	Допускается оформлять дату в последовательности: год, месяц, день месяца, например: 2009.06.05, а также использовать словесно-цифровой способ, например, 05 июня 2009 г. Проставлять ноль в обозначении дня месяца, если он содержит одну цифру, – обязательно.
	

	\item Правила оформления реквизита №8 «Справочные данные об организации».
	
	Справочные данные об организации (реквизит 9) включают в себя: почтовый адрес, номер телефона и другие сведения по усмотрению организации.1 ГОСТ Р 6.30-2003 "Унифицированные системы документации. Унифицированная система организационно-распорядительной документации. Требования к оформлению документов" не лимитирует состав и объем сведений. При оформлении этого реквизита могут также указываться номера факсов, телексов, счетов в банке, адрес электронной почты, адрес Интернет-сайта (web-страницы в Интернете).
	В составе справочных данных об организации указываются: код организации, основной государственный регистрационный номер (ОГРН) юридического лица, идентификационный номер налогоплательщика/код причины постановки на учет (ИНН/КПП).
	Почтовый адрес в справочных данных об организации должен оформляться в соответствии с пунктом 22 Правил оказания услуг почтовой связи, утвержденных приказом Минкомсвязи России от 31 июля 2014 г. № 234. Реквизиты адреса пишутся в следующем порядке:
	название улицы, номер дома, номер квартиры;
	название населенного пункта (города, поселка и т.п.);
	название района;
	название республики, края, области, автономного округа (области);
	почтовый индекс.
	

	\item На документы на каких носителях распространяются положения ГОСТ Р 7.0.97-2016
	
	Распространяются на документы на бумажном и электронном носителях

	\item Что такое трафаретный текст?
	
	Имеющий заранее отпечатанный стандартный текст (часть текста) и дополняющее его конкретное содержание

	\item В каких документах используют трафаретный текст?
	
	Трафаретный текст используется в документах с часто повторяющимся текстом или его частью, таких как анкета, справка, заявление, т. е. в таких документах, которые оформляются регулярно.

	\item Может ли документ с трафаретным текстом быть заполнен вручную?
	
	Изменяемая часть может быть заполнена вручную.

	\item Каким шрифтом набирается постоянная информация в шаблоне? Почему?
	
	Машинописному тексту соответствует шрифт Times New Roman 14 размера, этим шрифтом набирается постоянная информация шаблона для удобства ввода переменной информации на пишущей машинке

	\item Какова последовательность создания шаблонов?
	
	Программы типа MS Word предусматривают создание документов на основе типовых форм – «шаблонов», содержащих как заранее заданные элементы оформления, так и трафаретные тексты.
	При создании шаблона необходимо правильно разместить реквизиты каждого вида документа. 
	

	\item На каком бланке и с какими реквизитами в соответствии с УСОРД оформляется штатное расписание?
	
	Для составления штатного расписания применяется унифицированная форма № Т-3, утв. постановлением Госкомстата России от 05.01.2004 № 1 «Об утверждении унифицированных форм первичной учетной документации по учету труда и его оплаты».
	При составлении используются следующие реквизиты:
	1) Наименование организации
	2) Код организации
	3) Дата составления
	4) Период действия

	\item Кем визируется, подписывается и утверждается штатное расписание?
	
	Штатное расписание визируется руководителями подразделений, подписывается руководителем кадровой службы и главным бухгалтером, утверждается приказом (распоряжением), подписанным руководителем организации или уполномоченным им на это лицом.

	\item С чего начинается запись формулы в Excel?
	
	Все формулы начинаются со знака равенства (=)

	\item Как записывается в Excel формула вычитания содержимого двух ячеек?
	
	= Имя уменьшаемой ячейка – Имя вычитаемой ячейки (пр. C6-E6)

	\item Как посчитать сумму значений нескольких ячеек? Приведите пример.
	
	 =СУММ(A2:A10)
	=СУММ(A2:A10;C2:C10)
	

	\item Что включает текст резюме?
	
	Итоговый текст резюме должен представлять собой выжимку, из которой убрано все, что в принципе можно убрать без потери смысла: вводные слова, эпитеты, причастные и деепричастные обороты, лишние отглагольные прилагательные и существительные. Текст резюме включает в себя сведения об образовании и трудовой деятельности в обратном хронологическом порядке, место, на которое претендует составитель, специальные навыки, данные о составителе. 

	\item Каким образом оформляется трудовой договор?
	
	Итоговый текст резюме должен представлять собой выжимку, из которой убрано все, что в принципе можно убрать без потери смысла: вводные слова, эпитеты, причастные и деепричастные обороты, лишние отглагольные прилагательные и существительные.

	\item Обязательно составлять контракт при приеме на работу?
	
	Статья 68 Трудового кодекса РФ. Оформление приема на работу Прием на работу оформляется приказом (распоряжением) работодателя, изданным на основании заключенного трудового договора. Содержание приказа (распоряжения) работодателя должно соответствовать условиям заключенного трудового договора. Из этого следует, что контракт (трудовой договор) должен составляться обязательно.

	\item В каком случае работник пишет заявление об увольнении?
	
	Заявление на увольнение оформляется тогда, когда желание уволиться исходит от работника.

	\item Кто является адресатом при написании объяснительной записки?
	
	Адресат объяснительной записки (на чье имя она составляется, например, генеральный директор) и лицо, которому ее следует передать (например, секретарь или начальник кадровой службы).

	\item Какие документы включает конструкторская документация?
	
	Конструкторская документация – графические и текстовые документы, которые в совокупности или в отдельности определяют состав и устройство изделия и содержат необходимые данные для его разработки, изготовления контроля, эксплуатации и утилизации.
	
	Состав конструкторской документации регламентирован ГОСТами единой системы КД(ЕСКД), которым определены, виды и комплектность конструкторских документов на изделия всех отраслей промышленности: чертежи деталей, сборочный, общего вида, теоретический, габаритный, монтажный; чертеж-схема; спецификация, техническое описание, ведомости, пояснительная записка и др
	
	\item Какой документ является основным в технологической документации?
	
	Основной технологический документ – технологическая карта, которая дает подробное описание, и в которой приводятся расчеты всех технологических операций по изготовлению изделия. Технологические карты в практической деятельности руководителей и специалистов представлены следующими разновидностями:
	– операционной, фиксирующей отдельные производственные операции (сверление, шлифование, крепление, монтаж и т. п.);
	– общей (маршрутной) с последовательностью операций; цикловой с перечислением группы операций одного рабочего или одного цеха;
	– типового технологического процесса, содержащего сведения о средствах технологического оснащения и материальных нормативах для изготовления группы деталей и сборочных единиц.
	

	\item В каком случае оформляется титульный лист на комплект технологической документации?
	
	Первым листом комплекта технологических документов является титульный лист, который оформляется в соответствии с ГОСТ 3.1105. Обязательность применения ТЛ устанавливается на уровне отрасли или предприятия (организации).

	\item Что такое экспериментально-проектная документация? Привести примеры.
	
	?

	\item Реферат является научно-исследовательской работой?
	
	В соответствии с ГОСТ 7.32-2001 реферат является структурным элементом отчета о НИР

	\item Отчет по экспериментально-проектной работе является научно-
исследовательским документом?

Отчет по экспериментально-проектной работе является научно-исследовательским документом

	\item Каким ГОСТом (указать номер и наименование ГОСТа) устанавливаются правила оформления отчетов о научно-исследовательских работах?
	
	Структура и правила оформления отчетов о научно-исследовательских работах установлены ГОСТ 7.32-2001 Система стандартов по информации, библиотечному и издательскому делу. Отчет о научно-исследовательской работ

	\item Какому номеру УДК соответствует <<Компьютерные сети>>
	
	УДК 004.7

	\item Что такое <<документооборот>>?
	
	Документооборот – это движение документов в организации с момента их создания или получения до завершения исполнения: отправки и (или) направления в дело. Различают входящий, исходящий и внутренний документооборот.

	\item Какие документы называют входящими?
	
	Входящими являются документы, которые поступили на предприятие от внешних партнеров. Большинство входящих документов должны порождать соответствующие исходящие, причем в заранее установленные сроки. Сроки устанавливаются или нормативными актами, предписывающими то или иное время ответа на соответствующий входящий документ, или сроком исполнения, указанным непосредственно во входящем документе.

	\item Исходящий документ составляется только в ответ на входящий?
	
	Исходящими документами являются ответы организации на соответствующие входящие документы. Некоторая часть исходящих документов готовится на основе внутренних документов предприятия. Небольшое число исходящих документов может требовать поступления входящих документов.

	\item Что такое регистрационный номер в журнале регистрации?
	
	Регистрационный номер документа — цифровое или буквенно-цифровое обозначение, присваиваемое документу при его регистрации.
	Регистрационный номер и дата регистрации документа проставляются на подлиннике документа от руки в реквизитах бланка и на копии с визами, которая остается в организации и подшивается в дело, в соответствии с номенклатурой дел. Первый экземпляр отправляется адресату.
	

	\item На каких документах можно не проставлять штамп <<Вх. №>>?
	
	Штамп с текстом "ВХ. №\rule{3em}{.1pt}" применяется для штемпелевания документов. Штамп с сокращенным словом Входящий № (Вх.№) воспроизводится на документе принятом от заинтересованных лиц (лица), о чем на документе указывается входящий номер и дата принятия документа.

	\item В какой период года составляется номенклатура дел?
	
	в последнем квартале предшествующего года

	\item Каким должен быть первый раздел номенклатуры?
	
	служба документационного обеспечения управления (общий отдел)

	\item В каких подразделениях организации хранятся 4 экземпляра номенклатуры дел?

	\item Какой минимальный и максимальный срок хранения документов?

	\item Какие документы отправляют на государственное хранение?

	\item В соответствии с каким документом располагаются дела?

	\item Что такое опись дела?
	
	Опись дел – это архивный справочник, представляющий собой
	систематизированный перечень заголовков дел и предназначенный для
	раскрытия состава и содержания дел, закрепления их систематизации внутри
	фонда и учета дел.
	

	\item В каком случае оформляется лист-заместитель, а в каком карта-заместитель?
	
	Документы, уже включенные в дело, могут в течение года потребоваться
	работнику организации (учреждения). На их место закладывается лист-заместитель.
	Если требуется выдать целиком все дело, заполняется карта-заместитель.

	\item Можно ли через 2 года с момента заведения дела выдать отдельный документ из дела положив в него лист-заместитель?
	
	После делопроизводственного года выдача отдельных документов из дела
	не допускается. При необходимости во временное пользование может быть выдано дело целиком.

	\item На какой максимальный срок может быть выдано дело?
	
	Не более одного месяца.

\end{enumerate}

\end{document} % конец документа


\chapter{Основная часть}

Компания ООО <<Харита>>, в которой мне посчастливилось проходить практику, занимается разработкой и технической поддержкой программного обеспечения с акцентом на применение в финансовом секторе. Мне было поручено провести поверхностный аудит безопасности удалённой площадки одного из клиентов компании.

\begin{figure}
	\caption{Угрозы безопасности}
	\includegraphics[width=\textwidth]{images/main_scheme.png}
\end{figure}

\section{Аудит безопасности}

Для получения согласия клиента на проведения аудита было сформулировано сообщение следующего содержания:  <<Аудит позволяет оценить текущую безопасность функционирования информационной системы, оценить и прогнозировать риски, управлять их влиянием на бизнес-процессы фирмы, корректно и обоснованно подойти к вопросу обеспечения безопасности ее информационных активов, стратегических планов развития, маркетинговых программ, финансовых и бухгалтерских ведомостей, содержимого корпоративных баз данных В конечном
счете, грамотно проведенный аудит безопасности информационной системы позволяет добиться максимальной отдачи от средств, инвестируемых в создание и обслуживание системы безопасности
фирмы.>>

\section{Проверка векторов атаки}

\subsection{Доверенная загрузка}

SecureBoot\footnote{https://ru.wikipedia.org/wiki/Secure\_boot} — это программная технология, при помощи которой UEFI-совместимая прошивка может проверить подлинность исполняемых ей внешних компонентов (загрузчиков, драйверов и UEFI OptionROM'ов). Эти исполняемые компоненты должны быть подписаны ЭЦП, которая проверяется во время загрузки и в случае ее полного отсутствия, повреждения, отсутствия в списке доверенных (db) или присутствия в списке запрещенных (dbx) запуск соответствующего компонента не происходит. 

В проверяемых мной системах доверенная загрузка была включена.

\subsection{Сервис SSH}

OpenSSH\footnote{https://ru.wikipedia.org/wiki/OpenSSH} - свободно распространяемая версия семейства инструментов для удаленного управления компьютерами и передачи файлов с использованием протокола безопасной оболочки (SSH). Традиционные инструменты, используемые для этих функций, такие как telnet\footnote{https://ru.wikipedia.org/wiki/Telnet} и rcp\footnote{https://en.wikipedia.org/wiki/Berkeley\_r-commands}, незащищены и передают пользовательский пароль открытым текстом. OpenSSH предоставляет сервис на сервере и клиентские приложения для облегчения операций защиты, зашифрованного удаленного управления и передачи файлов, эффективно заменяя устаревшие инструменты.

\subsection{SELinux}

SELinux\footnote{https://ru.wikipedia.org/wiki/SELinux} — это система принудительного контроля доступа, реализованная на уровне ядра. Впервые эта система появилась в четвертой версии CentOS, а в 5 и 6 версии реализация была существенно дополнена и улучшена. Эти улучшения позволили SELinux стать универсальной системой, способной эффективно решать массу актуальных задач. Стоит помнить, что классическая система прав Unix применяется первой, и управление перейдет к SELinux только в том случае, если эта первичная проверка будет успешно пройдена.



\section{Устранение брешей}

\subsection{Сервис SSH}

В первую очередь следуют защититься от систематически повторяющихся попыток подбора пароля - bruteforce\footnote{https://en.wikipedia.org/wiki/Brute-force\_attack}.

Сервис Fail2ban может смягчить эту угрозу при помощи правил, автоматически меняющих настройки брандмауэра iptables\footnote{https://ru.wikipedia.org/wiki/Iptables}; эти правила срабатывают в случае поступления определённого количества неудачных попыток входа. Этот инструмент позволяет защитить сервер от несанкционированного доступа без вмешательства системного администратора.

Инструмент Fail2ban не доступен в официальном репозитории CentOS, но его можно получить в EPEL. Пакет самого репозитория EPEL (Extra Packages for Enterprise Linux) можно добавить из официального репозитория CentOS.

\begin{lstlisting}
sudo yum install epel-release
\end{lstlisting}

Теперь можно установить пакет fail2ban:

\begin{lstlisting}
sudo yum install fail2ban
\end{lstlisting}

После завершения установки следует использовать инструмент systemctl, чтобы включить fail2ban.

\begin{lstlisting}
sudo systemctl enable fail2ban
\end{lstlisting}

Сервис fail2ban хранит настройки в каталоге /etc/fail2ban. В нём можно найти файл jail.conf, содержащий стандартные настройки.

Этот файл перезаписывается при обновлении пакета Fail2ban, потому его редактировать нельзя. Вместо этого нужно создать новый файл по имени jail.local. Значения в файле jail.local будут переопределять jail.conf.

Файл jail.conf содержит раздел [DEFAULT], после которого следует раздел для индивидуальных сервисов. Файл jail.local может переопределить любое из этих значений. Файлы применяются в алфавитном порядке:

\begin{enumerate}
	\item /etc/fail2ban/jail.conf
	\item 	/etc/fail2ban/jail.d/*.conf,
	\item 	/etc/fail2ban/jail.local
	\item 	/etc/fail2ban/jail.d/*.local,
\end{enumerate}

Теперь изменим настройки sshd. Будет редактировать файл nano /etc/ssh/sshd\_config.

Поменяем порт для подключения по умолчанию.

\begin{lstlisting}
Port 1337
\end{lstlisting}

Превентивно заставим использовать протокол второй версии.

\begin{lstlisting}
Protocol 2
\end{lstlisting}

Отменим удалённое подключение для root пользователя.

\begin{lstlisting}
PermitRootLogin no
\end{lstlisting}

\subsection{SELinux}

SELinux имеет три основных режим работы, при этом по умолчанию установлен режим
\textbf{Enforcing}. Это довольно жесткий режим, и в случае необходимости он может быть изменен на более удобный для конечного пользователя.

\textbf{Enforcing}: Режим по умолчанию. При выборе этого режима все действия, которые каким-то образом нарушают текущую политику безопасности, будут блокироваться, а попытка нарушения будет зафиксирована в журнале.

\textbf{Permissive}: В случае использования этого режима, информация о всех действиях, которые нарушают текущую политику безопасности, будут зафиксированы в журнале, но сами действия не будут заблокированы.

\textbf{Disabled}: Полное отключение системы принудительного контроля доступа.

Текущий статус можно посмотреть командой \textbf{sestatus}.

\begin{lstlisting}
SELinux status:                 enabled
SELinuxfs mount:                /selinux
Current mode:                   enforcing
Mode from config file:          enforcing
Policy version:                 21
Policy from config file:        targeted
\end{lstlisting}

SELinux далеко не так страшен, как о нем говорят. Система хоть и сложна в понимании, но невероятно логична и удобна в сопровождении, а имеющиеся средства управления позволяют очень точно диагностировать проблемы и легко их устранять.



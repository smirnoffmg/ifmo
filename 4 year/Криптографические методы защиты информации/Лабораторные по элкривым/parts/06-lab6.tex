\section{Проверка ЭЦП на основе эллиптических кривых}

Цель работы: проверить подлинность ЭЦП $(r,s)$ для сообщения с известным значением хэш-свертки $e$, зная открытый ключ проверки подписи $Q$. Используется эллиптическая кривая $E751(−1,1)$ и генерирующая точка $G = (562, 89)$ порядка $n = 13$.

\begin{table}[H]
	\centering
	\begin{tabular}{|c|c|}
		\hline 
		Вариант & 7 \\ 
		\hline 
		e & 4 \\ 
		\hline 
		Q & (384, 475) \\ 
		\hline 
		(r, s) & (11, 9) \\ 
		\hline 
	\end{tabular} 
\end{table}


Проверка подписи начинается с проверки условий $ 1 \leq r \leq n−1, 1 \leq s \leq n−1 $.

$1 \leq 11 \leq 12, 1 \leq 9 \leq 12$.	


Вычисляем $v = s^{-1} \mod(n) = 9^{-1} \mod(13) = 3$

Вычисляем $u_1 = e \cdot v \mod(n) = 4 \cdot 3 \mod(13) = 12$

Вычисляем $u_2 = s \cdot v \mod(n) = 9 \cdot 3 \mod(13) = 1$

Находим точку $X=u_1\cdot G+u_2 \cdot Q = 12 \cdot (562, 89) + 1 \cdot (384, 475)$

 Найдем $2G = 1*G+1*G=(562, 89) + (562, 89) = (135, 669)$
\begin{equation*}
	\begin{aligned}
		\lambda &= \frac{ 3 \cdot 562^2 + (-1) }{2 \cdot 89} = \frac{947531}{178} = 947531 \cdot 178^{-1} = 520 \cdot 308\mod{751} = 197\mod{751} \\
		x &= 197^2 - 562 - 562 \mod{751} = 135\mod{751} \\
		y &= 197 \cdot (562 - 135) - 89\mod{751} = 669\mod{751}
	\end{aligned}
\end{equation*}

Найдем $4G = 2*G+2*G=(135, 669) + (135, 669) = (596, 318)$
\begin{equation*}
	\begin{aligned}
		\lambda &= \frac{ 3 \cdot 135^2 + (-1) }{2 \cdot 669} = \frac{54674}{1338} = 54674 \cdot 1338^{-1} = 602 \cdot 87\mod{751} = 555\mod{751} \\
		x &= 555^2 - 135 - 135 \mod{751} = 596\mod{751} \\
		y &= 555 \cdot (135 - 596) - 669\mod{751} = 318\mod{751}
	\end{aligned}
\end{equation*}

Найдем $8G = 4*G+4*G=(596, 318) + (596, 318) = (416, 696)$
\begin{equation*}
	\begin{aligned}
		\lambda &= \frac{ 3 \cdot 596^2 + (-1) }{2 \cdot 318} = \frac{1065647}{636} = 1065647 \cdot 636^{-1} = 729 \cdot 431\mod{751} = 281\mod{751} \\
		x &= 281^2 - 596 - 596 \mod{751} = 416\mod{751} \\
		y &= 281 \cdot (596 - 416) - 318\mod{751} = 696\mod{751}
	\end{aligned}
\end{equation*}

 Найдем $12G = 4*G+8*G=(596, 318) + (416, 696) = (562, 662)$
\begin{equation*}
	\begin{aligned}
		\lambda &= \frac{696-318}{416-596} = \frac{378}{-180} = 378 \cdot -180^{-1}\mod{751} = 378 \cdot 630\mod{751} = 73\mod{751} \\
		x &= 73^2 - 596 - 416 \mod{751} = 562\mod{751} \\
		y &= 73 \cdot (596 - 562) - 318\mod{751} = 662\mod{751}
	\end{aligned}
\end{equation*}

\textbf{X = (596, 433)}

Сравниваем значения $r$ и $x \mod n$, если они совпадают, следовательно, подпись действительная.
\begin{equation*}
	\begin{aligned}
		r &= 11 \\
		x \mod n &= 11
	\end{aligned}
\end{equation*}

Значия совпадают, следовательно,  подпись действительная.
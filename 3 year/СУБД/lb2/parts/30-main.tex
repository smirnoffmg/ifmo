\chapter{Введение}

\section{Цель работы}

Получение навыков манипулирования данными в БД при помощи операторов SQL.

\section{Теоретическая информация}

Особенности хранения данных в БД накладывает ограничения на манипулирование этими данными. Для эффективного решения задач, связанных с получением нужной информации из БД существует специальных язык структурированных запросов в БД – SQL. Все запросы оформляются в виде операторов SQL, обладающих собственным синтаксисом. В зависимости от реализации СУБД реализация SQL запросов может несколько отличаться, но базовые примитивы остаются неизменными. Наряду со стандартными операциями работы с данными – вставка, удаление, извлечение, обновление, SQL обладает расширенными возможностями по работе с данными, позволяющими получать данные в соответствии с разнообразными условиями. Владение навыками работы с БД при помощи SQL позволяет эффективно решать задачи по систематизации, поиску, хранению информации в СУБД.
В качестве СУБД, используемой в лабораторной работе, предполагается MySQL. 

\chapter{Ход работы}



\section{Наполнить таблицы базы данных при помощи операторов INSERT}

\lstinputlisting{src/1.in}


\section{Обновить записи в одной таблице на основании записи из другой}

\lstinputlisting{src/2.in}

\section{Вывести часть столбцов из таблицы}

\lstinputlisting{src/4.in}

\section{Вывести несколько записей из таблицы, используя условие ограничения}

\lstinputlisting{src/5.in}

\section{Сделать декартово произведение двух таблиц}

\lstinputlisting{src/6.in}

\section{Вывести записи из таблицы на основании условия ограничения, содержащегося в другой таблице}

\lstinputlisting{src/7.in}

\section{Применить функции агрегирования к выводимым записям }

\lstinputlisting{src/8.in}

\section{Вывести записи из таблицы, используя сортировку от большего к меньшему}

\lstinputlisting{src/9.in}

\section{Вывести записи из таблицы, используя сортировку от меньшего к большему с ограничением количества выводимых строк}

\lstinputlisting{src/10.in}

\section{Произвести агрегирование выводимых записей по одному из полей }

\lstinputlisting{src/11.in}


\chapter{Вывод}

В рамках данной работы были получены первичные навыки управления данными в БД при помощи SQL.
\section*{Введение}%
\addcontentsline{toc}{section}{Введение}%

В данной работе будут рассмотрены методы сжатия информации. Актуальность проблем сжатия не уменьшается с ростом объёма носителей информации и скорости передачи данных, поскольку одновременно происходит и рост объёма передаваемой информации. 

Все методы сжатия можно разделить на две большие группы: сжатие с потерями и сжатие без потерь. Сжатие без потерь применяется в тех случаях, когда информацию нужно восстановить с точностью до бита. Такой подход является единственно возможным при сжатии, например, текстовых данных. 

В некоторых случаях, однако, не требуется точного восстановления информации и допускается использовать алгоритмы, реализующие сжатие с потерями, которое, в отличие от сжатия без потерь, обычно проще реализуется и обеспечивает более высокую степень архивации. 

Целью данной работы будет рассмотрение методов сжатия с потерями и без потерь, а предметом - конкретные методы сжатия из обоих групп.

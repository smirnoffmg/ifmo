\section{Введение}

\subsection{Цель работы}

Цель работы - ознакомление с методами экспериментального исследования фоторезистора. Определение основных параметров и характеристик в различных режимах работы.


\subsection{Краткие теоретические сведения}

Фоторезистор - полупроводниковый прибор с однородной
(гомогенной) структурой, применяется в оптоэлектронике в качестве фотоприемника с внутренним фотоэффектом. Электрическое сопротивление фоторезистора меняется под действием светового потока.

В отличие от фотоприемников с внешним фотоэффектом в фоторезисторах можно получать значительные по величине фототоки и значительные выходные мощности. Это обстоятельство, а также высокая фоточувствительность, присущая фоторезисторам, и простота их конструкции определила их широкое использование в цепях фотоэлектрической автоматики, оптоэлектроники, пороговых приемников и пр.

Наиболее распространенными материалами для фоторезисторов являются соединения кадмия с серой, селеном, теллуром, чувствительные к видимой области спектра, кремний с добавками золота или цинка для ближнего инфракрасного излучения, соединения свинца с серой и селеном, германия, сурьмянистого индия для длинноволнового инфракрасного излучения.
\section{Атака на алгоритм шифрования RSA методом бесключевого чтения}

\textbf{Цель работы}: изучить атаку на алгоритм шифрования RSA посредством метода бесключевого чтения.

Дано:

$N = 
\seqsplit{12999930287191924431376539346299017010017110757740921516255625276736938341707759829697430675432898253598392520230520704679160980923414190358178622745742624474755389622466492660886890337924536205112736638940978976344972253628353579832538666892892665706708885193359305409807981966665666702224514486914538968870073023511710369161207889173238412011206880810244366176665150381905574765282393727109561493216690612279226503572974152890921340588167058405144084456523945787108461024171789972198504480081600349216283801717806994595190979350072982046379512550434705815641316760415616932877994091207418811935739845349184344241913}$

$e_1 = 1361953$

$e_2 = 1360027$

$C_1 = \seqsplit{12088343252205973698177879495532558160628407159919598897900358404178513416771114151553436558862570513528584259439782876460111075107403457975420713763599591561617281114708696859472465538975319877743512559006059886374736532091867961955471039099167850332657033795843954349543953032258491203901363222148491195613271905897998400944808165294965913009501612898001171541503440163194822580978042842810607817973439673938102475557024923838836032764707128565956760156893134148300723356238802901468313942361777600847182124122062344034877306087933560979545981654480134659927243673325694450087057858249060039863739686707181557266325}$

$C_2 = 
\seqsplit{363259649963520880272701383694947563320727632718468461195867622928915974342916415113694952302438767448890295157955288129370535694134081572468179363645046558988116451960372069411615883271246572778366814914120801365227788543512211620729324558495427258618372186072305765009838214305302715034245183523214128085228212226911018781118139463294676344735434728488692747458456411691653429430840421600113305878973232852200010542512528272763513962315011931157559677110833867088163283176788434226541575627440732209771108556989122914813897696630323521862316151299817288808932897797684509145949791493289998100323724505141715276442}$

Решаем уравнение $e_1 cdot r – e_2 \cdot s = \pm 1$ 

$A = e_1$

$B = e_2$

$A ∙ D – B ∙ C = N$

$C = s = 140721$
$D = r = 140522$

$A ∙ D – B ∙ C = – 1$

Производим дешифрацию: $c_1$ возводим в степень $r$, а $c_2$ – в степень $–s$ по модулю $N$. 

$c_1^r = \seqsplit{12212443334325089604301103215893734943220381756214881084473343919831705315107791175158097028101441631606001800304951331636203194244736999831017762184505914192481196149953051361119907357284814170356657007771697057558809163936113504819554364078092811676290638758929901622170727992517600141008051016416752098253410415158246159437355898370856529351346114959894364075022306118892019187170871096497393176414856245609285982840137455287294307324013401628714337774040807431207440180870753535734077459334516661746805129998612086716823911899929089175442988191223621327117071331607171588708540505332158809054001388485431781785030}$

$c_2^{–s} =
\seqsplit{11031378503092292711233134511962721713183869464346023659864462379023978034133076057222785405722929251844473674675187104620095083578561801510295070938721233919543850135960636899247170100019446343366419871739174446639974989075063562505670106728130912582671445965513332070109807214861292597755146939668991252052871935444897081567161440546481274282978023890124008672939181492323092526044483674164248571168501444700624498586706594080758249308560141689236535054362760496051529898621979790840341602624280470459919233969146647784011941020901287626782949857716596921107893341923220570492472920423972519893447823707834337739893}$

После этого перемножаем результаты и получаем, что $m^{e1 ∙ r – e2 ∙ s} = 
\seqsplit{134720084868506554980541850246513455310019194554755150892168249025843832381225768525286389204901997698915639869905667576646213405858823105606931691660987132839119585782584000019730782111004680659598512547483428155440658624680483124542727142764526682040610069960091267641614464744172615577812346992924117190743242034660329183833607807908008666160962721991253655044117881621519394514314955417466872235270612210682253250336739998123184522517848284050695585387360021775271484626533480019223790316171656556655500015271508977247434621209504463214884902633060485731677695132033654213936519433611710474036950126970134031711989814134199867883675093769513088239973167431033379627193929598933820481537115712483218521935638188609095356113715108712166094022716577980385354543805915253451641643165810508891392984313691956813804132899364628286049764104198260046328582399227672818115806911372190930387350812308582719590326784908964586890874617103242207353601782276777698420816945283460038026318793371247996280474910078757050113715083527391668766691488283273014178013153022907494520890943475064061235282174711610682656964807028249663607113330255898091259700769663378146539824148420271866664000906969599762574629192119425964473396238876963895381201790}$

Далее находим обратное значение по модулю: $(m^{– (e1 ∙ r – e2 ∙ s)} mod N) =
\seqsplit{1752652383401137530145990819341280299162365498259582112618057293178517228009796498140634453383353927596396317448697170359966217633968697534107385048388208019312951076510427991896407754755207130549401218037374833776972246651238777034361571728223612839216677563917796778997596491182561328772257772769607436131182504815775825424265311863526779379483318197505045696417568940208361917046444415191835412524437761677823496205223586990355807338279483286814827294248378908365821463628032085879844979547761068945288279079989733537211182693031741449912840429330570910491367937170877329050530345083237926614247927967604575}$

Ответ: "информация - сведения (сообщения, данные) независимо от формы их представления; информационная система - совокупность содержащейся в базах данных информации и обеспечивающих ее обработку информационных технологий и технических средств"




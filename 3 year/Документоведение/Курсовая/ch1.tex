\section{Нормативно-правовая база}

Правовую основу электронного документооборота составляет обширный массив нормативно-правовых актов и методических документов. Источниками юридического регулирования деятельности с электронными документами являются Конституция РФ, специальные законы в области информации и информатизации, отраслевые законы, указы Президента РФ и Постановления Правительства РФ, акты министерств и ведомств, нормативно-правовые акты субъектов Федерации, а также международные нормативные акты.

Правовые аспекты, регламентирующие использование электронных технологий в документообороте, относятся к особой, интенсивно развивающейся отрасли современного права, так называемому информационному законодательству.

Общероссийская нормативно-правовая база в области регулирования электронного документооборота, архивного дела включает в себя следующие документы:

\begin{itemize}
\item Государственная система документационного обеспечения управления (ГСДОУ). В этом документе собраны основные принципы и нормы, определяющие общие требования к документационному обеспечению управления и обеспечения работы с документами в организациях.

\item Федеральный закон «Об архивном деле в Российской Федерации» от 22.10.2004 г. № 125-ФЗ. Данный закон регулирует отношения в сфере организации хранения, комплектования, учета и использования архивных документов.

\item Федеральный закон «О порядке рассмотрения обращений граждан Российской Федерации» от 02.05.2006 г. № 59-ФЗ. Настоящим Федеральным законом устанавливается порядок рассмотрения обращений граждан государственными органами, органами местного самоуправления и должностными лицами, в том числе обработка подобных обращений в электронном виде.

\item Федеральный закон «Об информации, информационных технологиях и о защите информации» от 27.07.2006 №149-ФЗ . Дает понятие электронного документа и смежных с ним областей.

\item Федеральный закон «Об организации предоставления государственных и муниципальных услуг» от 27 июля 2010 г. № 210-ФЗ. Данный закон регулирует возникающие в связи с предоставлением государственных и муниципальных услуг соответственно федеральными органами исполнительной власти, органами государственных внебюджетных фондов, исполнительными органами государственной власти субъектов Российской Федерации, а также местными администрациями и иными органами местного самоуправления, осуществляющими исполнительно-распорядительные полномочия.

\item Федеральный закон «Об электронной подписи» (от 06.04.2011 г. №63-ФЗ). Закон прописывает правила применения электронной подписи.

\end{itemize}


Конституция РФ \cite{constitution} - важнейший законодательный акт Российской Федерации - является своеобразным фундаментом правового регулирования информационных правоотношений. В настоящее время Конституция РФ является нормативно-правовым актом прямого действия. По нашему мнению, это повышает ее значение и роль в регулировании делопроизводства и документооборота. При использовании информационно-документационных технологий необходимо учитывать ряд норм и положений Основного закона РФ. В частности, согласно ст.71 Конституции, информация и связь находятся в ведении Российской Федерации. Гарантируется свобода массовой информации. Каждый имеет право свободно искать, получать, передавать, производить и распространять информацию любым законным способом (ст.29). Однако сбор, хранение, использование и распространение информации о частной жизни лица без его согласия не допускаются. Органы государственной власти и органы местного самоуправления, их должностные лица обязаны обеспечить каждому возможность ознакомления с документами и материалами, непосредственно затрагивающими его права и свободы, если иное не предусмотрено законом (ст.24).

Вопросы работы с электронными документами также затрагиваются в нормативных правовых актах, посвященным отдельным сферам правового регулирования: гражданское, административное, уголовное, уголовно-процессуальное, трудовое, налоговое и другое законодательство Российской Федерации. Например, в Гражданском кодексе России признается принципиальная возможность использования электронных документов и электронно-цифровой подписи в гражданско-правовых отношениях. Согласно Гражданскому кодексу, информация представляет собой объект гражданских прав, а электронные документы могут закреплять гражданские права и обязанности \cite{grcodex}. Так, статья 160 Гражданского кодекса предполагает возможность заключения сделок в электронной форме.

Уголовный кодекс Российской Федерации предусматривает ответственность за преступления в сфере компьютерной информации: неправомерный доступ к информации; создание, использование и распространение вредоносных программ для ЭВМ; нарушение правил эксплуатации ЭВМ, систем ЭВМ или их сетей \cite{ugcodex}. Он также устанавливает наказание за уничтожение документов, имеющих историческую или культурную ценность5.

Кодекс Российской Федерации об административных правонарушениях\cite{admcodex}, Уголовно-процессуальный кодекс\cite{upcodex}, Арбитражный процессуальный кодекс и Гражданский процессуальный кодекс содержат положения, позволяющие рассматривать документы в электронном виде в качестве письменных (вещественных) доказательств.

Отраслевые кодексы также содержат положения, обеспечивающие работу с электронными документами в соответствующих отраслях и сферах. Так, Налоговый кодекс закрепляет правило, согласно которому налоговая декларация в случаях, установленных настоящим Кодексом, может представляться на дискете или ином носителе, допускающем компьютерную обработку. Порядок представления налоговой декларации в электронном виде определяется Министерством финансов Российской Федерации. Кодекс также закрепляет право вести регистры налогового учета в электронном виде и (или) любых машинных носителях. Таможенный кодекс разрешает, например, представлять таможенную декларацию в форме электронного документа.

Федеральный закон «Об информации, информатизации и защите информации»\cite{informlaw} является основным, фундаментальным законодательным актом в структуре информационного права, который регулирует отношения возникающие при: 
\begin{itemize}
	\item «осуществлении права на поиск, получение, передачу, производство и распространение информации;

	\item применение информационных технологий;

	\item обеспечение защиты информации».
\end{itemize}

В данном законе содержатся определения важнейших терминов в области информационного и документационного обеспечения управления. При рассмотрении автоматизированных информационных систем (АИС) представляют интерес трактовки понятий «документированная информация » (это «зафиксированная на материальном носителе путем документирования информация с реквизитами, позволяющими определить такую информацию или в установленных законодательством Российской Федерации случаях ее материальный носитель»), «информация» (это «сведения (сообщения, данные) независимо от формы их представления»), «информационная система» («совокупность содержащейся в базах данных информации и обеспечивающих ее обработку информационных технологий и технических средств»), «доступ к информации» («возможность получения информации и возможность ее использовать»), «электронное сообщение» («информация переданная или полученная пользователем информационно-телекоммуникационной сети») и ряда других. Отметим, что термин «информация» определяется в законе как «сведения (сообщения, данные) независимо от формы их представления» .

Законом установлен правовой режим документирования информации:

1) документирование информации является обязательным условием включения ее в информационные ресурсы, а порядок документирования информации нормативно устанавливается органами государственной власти, ответственными за организацию делопроизводства, стандартизацию документов и их массивов, безопасность РФ1;

2) документ, полученный из автоматизированной информационной системы, приобретает юридическую силу после его подписания должностным лицом в порядке, определяемом законодательством России. Закон также допускает использование электронной цифровой подписи для подтверждения юридической силы документов при работе с информацией в электронном виде ;

3) за правонарушения при работе с документированной информацией организации их должностные лица должны нести ответственность в соответствии с законодательством РФ и ее субъектов15.

Таким образом, данный закон допускает и регулирует использование юридически значимых документов в электронном виде. Кроме того, законом установлены категории информации по уровню доступа, определено понятие «информация о гражданах (персональные данные)» (это сведения о фактах, событиях и обстоятельствах жизни гражданина, позволяющие идентифицировать его личность»), регламентирован правовой режим персональных данных. Эти положения закона также необходимо учитывать при создании систем электронного документооборота.

Новый Федеральный закон «О связи» \cite{svyazlaw} «регулирует отношения, связанные с созданием и эксплуатацией всех сетей и сооружений связи, оказанием услуг электросвязи и почтовой связи на территории РФ и на находящихся под юрисдикцией РФ территориях», «в отношении операторов связи, осуществляющих свою деятельность за пределами территории РФ в соответствии с правом иностранных государств, ФЗ применяется только в части регулирования порядка выполнения работ и оказания ими услуг связи на находящихся под юрисдикцией РФ территориях». В соответствии со ст.2 данного закона под сетью связи понимается «технологическая система, включающая в себя средства и линии связи и предназначенная для электросвязи и почтовой связи», «электросвязь - любые излучение, передача или прием знаков, сигналов, письменного текста или сообщений любого рода по радиосистеме, проводной, оптической и другим электромагнитным системам», «средства связи - технические и программные средства, используемые для формирования, приема, обработки, хранения, передачи, доставки сообщений электросвязи или почтовых отправлений, а также иные технические и программные средства, используемые при оказании услуг связи или обеспечении функционирования сетей связи».

Федеральный закон «Об участии в международном информационном обмене»\cite{barterlaw} обеспечивает правовое регулирование отношений в области передачи информации за пределы России и получении ее извне. Он определяет термин «информационные процессы» как «процессы создания, сбора, обработки, накопления, хранения, поиска, распространения и потребления информации».

Организация юридически значимого электронного документооборота стала возможной с использованием технологии электронной цифровой подписи (ЭЦП). Базовыми нормативно-правовыми актами, регулирующими первоначальные основы правового регулирования ЭЦП, являются Гражданский кодекс РФ (ст. 160), а также Федеральный закон " Об информации, информационных технологиях и о защите информации" (ст.11). Так, как уже было указано выше, ст. 160 Гражданского кодекса допускает использование ЭЦП при совершении сделок. Соответственно, федеральный закон "Об информации, информационных технологиях и о защите информации " определяет юридическую силу электронного документа только после подписания его ЭЦП, а также юридическая сила электронной цифровой подписи признается при наличии в автоматизированной информационной системе программно-технических средств, обеспечивающих идентификацию подписи, и соблюдении установленного режима их использования. Право удостоверять идентичность электронной цифровой подписи осуществляется на основании лицензии (ст.5).

Федеральным законом «Об электронной цифровой подписи»\cite{ecplaw} урегулирована технология проставления (порядок создания и применения) той разновидности электронной подписи, которая использует метод ассиметричной криптографии (открытого ключа) - электронной цифровой подписи (ЭЦП). Следует отметить, что действие закона распространяется только на документы, образующиеся при совершении гражданско-правовых сделок.

Федеральный закон «О бухгалтерском учете» допускает создание первичных и сводных учетных документов на бумажных и машинных носителях информации. В последнем случае организация должна изготовлять копии таких документов на бумажных носителях для других участников хозяйственных операций, а также по требованию органов, осуществляющих контроль.

Федеральный закон «Об архивном деле в Российской Федерации» гласит, что в состав Архивного фонда Российской Федерации включаются архивные документы независимо от способа их создания, вида носителя, в том числе электронные документы. Это означает, что электронные документы подлежат организации, упорядочению и сохранению в пределах сроков, установленных перечнями и другими нормативными актами наравне с традиционными30. Поэтому экспертизе ценности документов подлежат все документы на носителях любого вида, находящиеся в федеральной собственности, собственности субъекта Российской Федерации или муниципальной собственности.


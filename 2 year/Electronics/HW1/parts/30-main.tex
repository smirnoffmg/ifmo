\chapter{Расчёты}

    \section{Расчёт выпрямителя без фильтра}
    
    Амплитудное значение напряжения:
    \[
    U_{IN\_MAX} = U_{L\_RMS} \cdot \sqrt(2) = 250 \cdot \sqrt(2) = 353.55
    \]
    
Среднее напряжение на нагрузке, В:
\[
U_{L\_AVG}=0.45 \cdot U_{IN\_RMS} = 0.45 \cdot 250 = 112.5
\]

Действующее значение напряжения на нагрузке, В:

\[
U_{L\_RMS} = \frac{U_{IN\_MAX}}{2} = 176.78
\]

Средний ток нагрузки, А:

\[
I_{L\_AVG} = \frac{0.45 \cdot U_{IN\_RMS}}{R_L}= \frac{112.5}{300} = 0.375
\]

Действующее значение тока нагрузки, А:

\[
I_{L\_RMS} = \frac{U_{IN\_MAX}}{2 \cdot R_L}=\frac{353.55}{2 \cdot 300} = 0.589
\]

Действующее значение переменной составляющей напряжения нагрузки, В:

\[
U_{L\sim RMS} = \sqrt{U_{L\_RMS}^2-U_{L\_AVG}^2} = \sqrt{176.78^2-112.5^2}=136.33
\]

Средний ток диода, А:

\[
I_{VD}=I_{L\_AVG} = 0.375
\]

Максимальный ток диода, А:
\[
I_{VD\_MAX}=\frac{U_{IN\_MAX}}{R_L}=\frac{353.55}{300}=1.18
\]

Максимальное обратное напряжение на диоде, В:

\[
U_{VD\_MAX}=U_{IN\_MAX}=353.55
\]


Был выбран диод \textbf{S2GA} со следующими параметрами:

\[
I_{VD\_MAX} = 1,5 \text{А} ;
U_{VD\_MAX} = 400 \text{В}
\]

\section{Расчёт выпрямителя с фильтром}
Сопротивление диода, Ом:
\[
r_{VD} = \frac{0.92-0.61}{1.18}=0.26
\]

Выходное сопротивление источника, Ом:
\[
r_{OUT\_U\_IN}=0
\]

Входное сопротивление выпрямителя, Ом:
\[
r_{IN}=r_{VD}+r_{OUT\_U\_IN}=0.26
\]

Угол открытия диода, рад:
\[
\theta=2 \cdot \sqrt[3]{3 \cdot \pi \cdot \frac{r_{IN}}{R_L}}=2 \cdot \sqrt[3]{3 \cdot \pi \cdot \frac{0.26}{300}}=0.4
\]


Среднее напряжение на нагрузке, В:

\[
U_{L\_AVG}=U_{IN\_MAX} \cdot \cos (\frac{\theta}{2})= 353.55 \cdot \cos(\frac{0.4}{2})=346.35
\]

Средний ток нагрузки, А:

\[
I_{L\_AVG}=\frac{U_{L\_AVG}}{R_L}=\frac{343.46}{300}=1.145
\]

Средний ток диода, А:
\[
I_{VD}=\frac{U_{IN\_MAX}}{R_L} \cdot \cos \left(\frac{\theta}{2}\right)=\frac{353.55}{300} \cdot \cos (\frac{0.4}{2})=1.15
\]

Максимальный ток диода, А:
\[
I_{VD\_MAX}=\frac{U_{IN\_MAX}-U_{L\_AVG}}{r_{IN}}=\frac{500-242.84}{0.4}=27.38
\]

Ток диода при включении схемы, А:
\[
I_{VD\_ON}=\frac{U_{IN\_MAX}}{r_{IN}}=\frac{500}{0.44}=1344.64
\]

Максимальное обратное напряжение на диоде, В:
\[
U_{VD\_MAX}=U_{IN\_MAX}+U_{L\_AVG}=353.55+346.35=699.9
\]


Допустимая величина пульсаций, В:
\[
\Delta U_B = 1
\]

Коэффициент пульсаций:
\[
K_p = \frac{\Delta U_B}{U_{L\_AVG}\cdot  2 \cdot \sqrt{2}}=\frac{1}{346.45 \cdot 2 \cdot \sqrt{2}}=0.001021
\]


Выбор ёмкости, Ф:
\[
C=\frac{I_{L\_AVG}}{\omega \cdot \Delta U_B} \cdot (2 \cdot \pi - \theta)= \frac{1.145}{377} \cdot (2 \cdot 3.14 - 0.4)=0.018
\]

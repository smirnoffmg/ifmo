\section{Сертификация, лицензирование, аттестация. Ключевые особенности.}

\textbf{Сертификация} - деятельность по независимому подтверждению соответствия характеристик продукта, услуги или системы требованиям стандартов или иных нормативных документов.  Частный случай - сертификация средств защиты информации по требованиям безопасности информации.

Ключевые особенности сертификации:

\begin{itemize}
	\item Контроль соответствия формализованным требованиям
	\item Достоверность и повторяемость результатов, их документальное оформление
	\item Независимый характер сертификации: проводит третья сторона
	\item Может быть добровольной и обязательной
	\item Может проводиться в разных системах сертификации
\end{itemize}

\textbf{Лицензирование}  - мероприятия, связанные с предоставлением лицензий, переоформлением документов, подтверждающих наличие лицензий, приостановлением и возобновлением действия лицензий, аннулированием лицензий и контролем лицензирующих органов за соблюдением лицензиатами при осуществлении лицензируемых видов деятельности соответствующих лицензионных требований и условий.

\textbf{Аттестация} - официальное подтверждение наличия на объекте защиты необходимых и достаточных условий, обеспечивающих выполнение установленных требований руководящих документов по защите информации.

Под аттестацией объекта информатизации понимается комплекс организационно-технических мероприятий, в результате которых посредством специального документа - «Аттестата соответствия» подтверждается, что объект соответствует требованиям стандартов или иных нормативных документов по защите информации, утвержденных ФСТЭК (Гостехкомиссией) России.
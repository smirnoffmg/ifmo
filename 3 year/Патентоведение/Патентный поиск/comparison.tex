\section{Сравнение патентов}

В данной работе проведено патентное исследование на тему <<Способ повышения криптографической стойкости информации>> с глубиной поиска в 8 лет (2010 - 2018 гг.). В результате было найдено 10 патентов по данной теме. Поиск осуществлялся по патентам класса \textbf{H04K 1/00}

Патенты были разбиты на следующие группы:

\subsection{Повышение скрытности}

\begin{enumerate}
    
\item \textbf{RU 2 438 250 C1} "Повышения скрытности и надежности приема полезных радиосигналов"

\item \textbf{RU 2 446 588 C2} "Повышение скрытности передачи узкополосного информационного сигнала"

\item \textbf{RU 2 452 100 C1} "Повышение надежности синхронизации при доставке информации по многолучевым каналам с рассеянием"
 
\end{enumerate}



\subsection{Повышение криптографической стойкости ключа шифрования}

\begin{enumerate}
    \item \textbf{RU 2 422 885 C1} "Повышение уровня защиты передаваемой информации от несанкционированного доступа" 

    \item \textbf{RU 2 450 457 C1} "Разработка метода шифрования с четко определенным способом задания конечных некоммутативных групп, обладающего повышенной криптостойкостью при использовании относительно малых разрядностей чисел за счет значительного повышения вычислительной емкости алгоритма"

    \item \textbf{RU 2 474 064 C1} "Повышение степени защиты от несанкционированного доступа к информации, передаваемой по абонентским линиям и каналам связи, на всем тракте передачи от абонента местной телефонной связи до удаленного абонента, включенного в каналы связи "

    \item \textbf{RU 2 656 578 C1} "Обеспечение формирования ключей шифрования с повышенной криптостойкостью"

    \item \textbf{RU 2 613 845 C1} "Повышение стойкости сформированного ключа шифрования/дешифрования для сети связи, включающей трех корреспондентов, к компрометации со стороны нарушителя"

    \item \textbf{RU 2 656 713 C1} "Арифметическое кодирование с шифрованием избыточной двоичной информационной последовательности, обеспечивающее уменьшение требуемой скорости передачи зашифрованной информации"
\end{enumerate}


\subsection{Новый подход в формировании ключей шифрования}

\textbf{RU 2 488 965 C1} "Создание скрытого квантового ключа (СКК) на Земле с операцией передачи его на пункт направленной скрытой связи с низкоорбитальным космическим аппаратом (НА), с очередной операцией скрытой передачи СКК на другой НА, а с него заключительную прицельную скрытую операцию передачи на стационарные и движущиеся наземные, подземные, надводные, подводные, летательные объекты, нуждающиеся в быстрой постоянной смене СКК"



Из всех патентов наиболее интересным мне показался \textbf{RU 2 488 965 C1}, в том числе и потому, что других патентов на тему квантовой криптографии обнаружить не удалось.
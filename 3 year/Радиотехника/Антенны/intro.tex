\section*{Введение}
\addcontentsline{toc}{section}{Введение}%


Основные области использования радиоэлектроники — связь, телевидение, радиолокация, радиоуправление, радиоастрономия, а также системы определения государственной принадлежности, инструментальной посадки, радиоэлектронного противодействия, телеметрия и другие невозможны без применения антенн с различными характеристиками. В процессе развития антенн они усложнялись, появлялись принципиально новые их классы, расширялись выполняемые функции, и антенны зачастую превращались из простых взаимных устройств в сложные динамические системы, содержащие в большинстве случаев сотни, тысячи различных элементов.

Целью работы является изложение общих сведения о вибраторных антеннах, а также минимально необходимых основ теории антенн.


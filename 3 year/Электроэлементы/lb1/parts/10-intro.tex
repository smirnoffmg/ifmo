\section{Введение}

\subsection{Цель работы}

Цель работы - исследование зависимости емкости конденсаторов с различными диэлектриками от температуры и определение (ТКЕ) этих конденсаторов.

\subsection{Краткие теоретические сведения}

Одним из важнейших факторов, характеризующих внешние воздействия на электрические конденсаторы, является температура окружающей среды.

Температурная зависимость емкости конденсаторов характеризуется величиной температурного коэффициента емкости (ТКЕ) :

\[
\alpha_c=\frac{1}{C} \cdot \frac{dC}{dT}
\]

Если зависимость емкости от температуры носит линейный характер, то величину ТКЕ можно вычислить по формуле:

\[
\alpha_c = \frac{C_2-C_1}{C_1 (t_2-t_1)},
\]

где $\alpha_c$- температурный коэффициент емкости, $C_1$ - ёмкость при комнатной температуре, $C_2$ - ёмкость при измененной температуре 

При нелинейной зависимости ёмкости от температуры указанная формула даёт только среднее значение ТКЕ.

Характер зависимости емкости конденсатора от температуры
обычно определяется температурной зависимостью диэлектрической проницаемости применяемого в конденсаторе диэлектрика.

Кроме того, зависимость емкости от температуры обуславливается особенностями конструкции конденсатора и изменением его размеров при нагревании. Температурное расширение обкладок приводит к увеличению емкости, а увеличение толщины диэлектрика - к ее уменьшению.

В плоскости конденсатора с обкладками в виде квадрата со стороной $l$ емкость равняется:

\[
C = 0.0884 \cdot \frac{\varepsilon \cdot l^2}{d},
\]

где $C$ - ёмкость конденсатора, пФ, $\varepsilon$  - диэлектрическая проницаемость, $d$ - толщина диэлектрика, мм, $l$ - линейный размер, мм.

Дифференцируя это выражение по температуре, получим:

\[
\frac{dC}{dt} = 0.0884 \cdot \left( \frac{l^2}{d} \frac{d \varepsilon}{dt} + \frac{\varepsilon}{d} \cdot 2l \cdot \frac{dl}{dt} + \frac{\varepsilon \cdot l^2}{d^2} \frac{dd}{dt} \right)
\]

Разделив левую и правую части на выражение для ёмкости, имеем:

\[
\frac{1}{C} \frac{dC}{dt} = \frac{1}{\varepsilon} \frac{d \varepsilon}{dt} + \frac{2}{l} \frac{dl}{dt} - \frac{1}{d} \frac{dd}{dt},
\]

или

\[
\alpha_c=\alpha_\varepsilon + 2 \alpha_M-\alpha_d,
\]

где $\alpha_\varepsilon$ - температурный коэффициент диэлектрической проницаемости, $\alpha_M$ − коэффициент линейного расширения металлических обкладок, $\alpha_d$ − коэффициент линейного расширения диэлектрика.

Если конденсатор изготовлен способом металлизации диэлектрика, то расширение обкладок будет определяться не расширением металла, а расширением диэлектрика. В этом случае можно считать, что и формула принимает вид:

\[
\alpha_c=\alpha_\varepsilon + \alpha_d
\]
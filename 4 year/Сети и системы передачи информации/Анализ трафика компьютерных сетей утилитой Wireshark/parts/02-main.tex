\section{Ход работы}

\subsection{Анализ трафика утилиты ping}

\begin{figure}[H]
	\centering
	\includegraphics[width=0.7\linewidth]{"../Анализ трафика компьютерных сетей утилитой Wireshark/img/ping-1"}
	\caption{}
	\label{fig:ping-1}
\end{figure}

\begin{figure}[H]
	\centering
	\includegraphics[width=0.7\linewidth]{"../Анализ трафика компьютерных сетей утилитой Wireshark/img/ping-2"}
	\caption{}
	\label{fig:ping-2}
\end{figure}


\textbf{Имеет ли место фрагментация исходного пакета, какое поле на это указывает?}

Имеет при размере пакета более 1480 байт. На это указывают флаги в IP.

\textbf{Какая информация указывает, является ли фрагмент пакета последним или промежуточным?}

\textbf{Чему равно количество фрагментов при передаче ping-пакетов?}

\[
N = \lfloor \frac{B}{MTU} \rfloor + 1
\]

Здесь $N$ - количество пакетов, $B$ - размер пакета, $MTU$ - установленное MTU для сетевых устройств.

\textbf{Построить график, в котором на оси абсцисс находится
размер пакета, а по оси ординат – количество фрагментов, на
которое был разделён каждый ping-пакет.}

\begin{figure}[H]
	\centering
	\begin{tikzpicture}
	
	\pgfplotsset{
		scale only axis,
	}
	
	\begin{axis}[
	xmin=0, xmax=15000,
	ymin=0, ymax=12,
	xtick distance=3000,
	xlabel=Размер пакета,
	ylabel=Фрагменты,
	]
	\addplot[] 
	coordinates{
(100, 1)
(200, 1)
(300, 1)
(400, 1)
(500, 1)
(600, 1)
(700, 1)
(800, 1)
(900, 1)
(1000, 1)
(1100, 1)
(1200, 1)
(1300, 1)
(1400, 1)
(1500, 2)
(1600, 2)
(1700, 2)
(1800, 2)
(1900, 2)
(2000, 2)
(2100, 2)
(2200, 2)
(2300, 2)
(2400, 2)
(2500, 2)
(2600, 2)
(2700, 2)
(2800, 2)
(2900, 2)
(3000, 3)
(3100, 3)
(3200, 3)
(3300, 3)
(3400, 3)
(3500, 3)
(3600, 3)
(3700, 3)
(3800, 3)
(3900, 3)
(4000, 3)
(4100, 3)
(4200, 3)
(4300, 3)
(4400, 3)
(4500, 4)
(4600, 4)
(4700, 4)
(4800, 4)
(4900, 4)
(5000, 4)
(5100, 4)
(5200, 4)
(5300, 4)
(5400, 4)
(5500, 4)
(5600, 4)
(5700, 4)
(5800, 4)
(5900, 4)
(6000, 5)
(6100, 5)
(6200, 5)
(6300, 5)
(6400, 5)
(6500, 5)
(6600, 5)
(6700, 5)
(6800, 5)
(6900, 5)
(7000, 5)
(7100, 5)
(7200, 5)
(7300, 5)
(7400, 6)
(7500, 6)
(7600, 6)
(7700, 6)
(7800, 6)
(7900, 6)
(8000, 6)
(8100, 6)
(8200, 6)
(8300, 6)
(8400, 6)
(8500, 6)
(8600, 6)
(8700, 6)
(8800, 6)
(8900, 7)
(9000, 7)
(9100, 7)
(9200, 7)
(9300, 7)
(9400, 7)
(9500, 7)
(9600, 7)
(9700, 7)
(9800, 7)
(9900, 7)
(10000, 7)
(10100, 7)
(10200, 7)
(10300, 7)
(10400, 8)
(10500, 8)
(10600, 8)
(10700, 8)
(10800, 8)
(10900, 8)
(11000, 8)
(11100, 8)
(11200, 8)
(11300, 8)
(11400, 8)
(11500, 8)
(11600, 8)
(11700, 8)
(11800, 8)
(11900, 9)
(12000, 9)
(12100, 9)
(12200, 9)
(12300, 9)
(12400, 9)
(12500, 9)
(12600, 9)
(12700, 9)
(12800, 9)
(12900, 9)
(13000, 9)
(13100, 9)
(13200, 9)
(13300, 9)
(13400, 10)
(13500, 10)
(13600, 10)
(13700, 10)
(13800, 10)
(13900, 10)
(14000, 10)
(14100, 10)
(14200, 10)
(14300, 10)
(14400, 10)
(14500, 10)
(14600, 10)
(14700, 10)
(14800, 11)
(14900, 11)
	}; \label{plot_one}
	
	\end{axis}
	
	\end{tikzpicture}
\end{figure}

\textbf{Как изменить поле TTL с помощью утилиты ping?}

С помощью ключа \textbf{-T}.

\textbf{Что содержится в поле данных ping-пакета?}

Это переменная, зависит от ОС.


\subsection{Анализ трафика утилиты tracert (traceroute)}

\begin{figure}[H]
	\centering
	\includegraphics[width=0.7\linewidth]{"../Анализ трафика компьютерных сетей утилитой Wireshark/img/tracert-1"}
	\caption{}
	\label{fig:tracert-1}
\end{figure}


\textbf{Сколько байт содержится в заголовке IP? Сколько байт содержится в поле данных?}

В заголовке IP - 20 байт, в поле данных - 24.

\textbf{Как и почему изменяется поле TTL в следующих друг за другом
ICMP-пакетах tracert?}

Traceroute  отправляет по 3 ICMP-пакета, если  все запросы получают  ответ о невозможности дойти до адресата, то TTL увеличивается на 1, и происходит повторная отправка.

\textbf{Чем отличаются ICMP-пакеты, генерируемые утилитой tracert, от
ICMP-пакетов, генерируемых утилитой ping?}

У ping-пакетов Type=8 (Echo (ping) request),  а у traceroute Type=11 (Время жизни пакета истекло).

\textbf{Чем отличаются полученные пакеты «ICMP reply» от «ICMP error» и
зачем нужны оба этих типа ответов?}

\textbf{Что изменится в работе tracert, если убрать ключ $-d$? Какой
дополнительный трафик при этом будет генерироваться?}

Предотвращает попытки утилиты tracert разрешения IP-адресов промежуточных маршрутизаторов в имена. Для traceroute нужно использовать ключ $-n$.


\subsection{Анализ HTTP-трафика}

\begin{figure}[H]
	\centering
	\includegraphics[width=0.7\linewidth]{"../Анализ трафика компьютерных сетей утилитой Wireshark/img/http-1"}
	\caption{}
	\label{fig:http-1}
\end{figure}

\begin{figure}[H]
	\centering
	\includegraphics[width=0.7\linewidth]{"../Анализ трафика компьютерных сетей утилитой Wireshark/img/http-2"}
	\caption{}
	\label{fig:http-2}
\end{figure}


\subsection{Анализ DNS-трафика}

\begin{figure}[H]
	\centering
	\includegraphics[width=0.7\linewidth]{"../Анализ трафика компьютерных сетей утилитой Wireshark/img/dns-1"}
	\caption{}
	\label{fig:dns-1}
\end{figure}

\begin{figure}[H]
	\centering
	\includegraphics[width=0.7\linewidth]{"../Анализ трафика компьютерных сетей утилитой Wireshark/img/dns-2"}
	\caption{}
	\label{fig:dns-2}
\end{figure}

\textbf{Почему адрес, на который отправлен DNS-запрос, не совпадает с
адресом посещаемого сайта?}

Потому что сначала надо получить адрес сайта у ближайшего DNS-сервера.

\textbf{Какие бывают типы DNS-запросов?}

Существует 3 типа DNS-запросов:

\begin{enumerate}

	\item Рекурсивный: подобные запросы выполняют пользователи к резолверу. Собственно, это первый запрос, который выполняется в процессе DNS-поиска. Резолвером чаще всего выступает ваш интернет провайдер или сетевой администратор.
	
	\item Нерекурсивные: в нерекурсивных запросах резолвер сразу возвращает ответ без каких-либо дополнительных запросов на другие сервера имён. Это случается, если в локальном DNS-сервере закэширован необходимый IP-адрес либо если запросы поступают напрямую на авторитативные серверы, что позволяет избежать рекурсивных запросов.
	
	\item Итеративный: итеративные запросы выполняются, когда резолвер не может вернуть ответ, потому что он не закэширован. Поэтому он выполняет запрос на корневой DNS-сервер. А тот уже знает, где найти фактический TLD-сервер.
\end{enumerate}

\textbf{В какой ситуации нужно выполнять независимые DNS-запросы для
получения содержащихся на сайте изображений?}

Когда нет DNS-сервера, содержащего все записи хостов для всех изображений.


\subsection{Анализ ARP-трафика}

\begin{figure}[H]
	\centering
	\includegraphics[width=0.7\linewidth]{"../Анализ трафика компьютерных сетей утилитой Wireshark/img/arp-1"}
	\caption{}
	\label{fig:arp-1}
\end{figure}

\textbf{Какие МАС-адреса присутствуют в захваченных пакетах ARP-
протокола? Что означают эти адреса? Какие устройства они
идентифицируют?}

Адреса отправителя и получателя.


\subsection{Анализ трафика утилиты nslookup}

\begin{figure}[H]
	\centering
	\includegraphics[width=0.7\linewidth]{"../Анализ трафика компьютерных сетей утилитой Wireshark/img/nslookup-1"}
	\caption{}
	\label{fig:nslookup-1}
\end{figure}

\textbf{Чем различается трасса трафика в п.2 и п.4, указанных выше? }

В данном примере - только типом ответа (A для второго пункта, NS для четвертого). 

\textbf{Что содержится в поле «Answers» DNS-ответа?}

DNS-записи для запрошенного хоста.

\textbf{Каковы имена серверов, возвращающих авторитативный (authoritative) отклик?}

В данном примере их нет.

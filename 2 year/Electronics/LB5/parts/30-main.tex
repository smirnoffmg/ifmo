\chapter{Ход работы}

\textbf{Цель работы}: провести модельное исследование фильтра нижних частот по схеме Саллека-Ки с использованием выбранных элементов.

Исходные данные: условия (дано) из ДЗ и результаты расчетов

\begin{figure}[h!]
	\centering
	\caption{Схема фильтра частот}
	\includegraphics{images/1.png}
\end{figure}


\begin{figure}[h!]
	\centering
	\caption{Модель системы в OrCAD Capture}
	\includegraphics{images/2.png}
\end{figure}

\chapter{Результаты моделирования}

\begin{figure}[h!]
	\centering
	\caption{Входное напряжение (зеленый, В), выходное напряжение при 200 Гц (красный, В) }
	\includegraphics{images/3.png}
\end{figure}

\begin{figure}[h!]
	\centering
	\caption{Входное напряжение (зеленый, В), выходное напряжение при 500 Гц (красный, В)}
	\includegraphics{images/4.png}
\end{figure}

\begin{figure}[h!]
	\centering
	\caption{Входное напряжение (зеленый, В), выходное напряжение при 1000 Гц (красный, В)}
	\includegraphics{images/5.png}
\end{figure}

\begin{figure}[h!]
	\centering
	\caption{Входное напряжение (зеленый, В), выходное напряжение при 50 Гц (красный, В)}
	\includegraphics{images/6.png}
\end{figure}

\begin{figure}[h!]
	\centering
	\caption{АЧХ}
	\includegraphics{images/7.png}
\end{figure}


\chapter{Измерения в OrCAD Capture}

Максимальное выходное напряжение, В:   				          

\[
 U_{OUT\_MAX\_EXP}=7.9839
 \]

Минимальное выходное напряжение, В:

\[
U_{OUT\_MIN\_EXP}=-7.9839
\]

\chapter{Вычисление погрешностей}

Погрешность $U_{OUT\_MAX}$ при 200 Гц:
\[
\left| \frac{U_{OUT\_MAX\_EXP}-U_{OUT\_MAX}}{U_{OUT\_MAX}} \right|  =0,3\%
\]

Погрешность спада:

\[
\left| \frac{f_{EXP}-f}{f}\right| = 7.5 \%
\]

\chapter{Вывод}

В ходе выполнения лабораторной работы было был промоделирован фильтра нижних частот по схеме Саллека-Ки, построены графики изменения величин в OrCAD PSpice, измерены необходимые величины по этим графикам и рассчитаны погрешности.

Вычисленные погрешности не превышают отклонение в 10\% от расчетных значений, что свидетельствует о корректности выполнения работы и соответствии модели расчетам.




\section{Ход работы}

\section{IP-камера}

Для видеонаблюдения за домашним животным достаточно сетевой камеры, подключённой к интернету. У всех современных камер есть приложения для компьютера и мобильного телефона, с помощью которых можно просматривать изображение из любой точки мира.

Некоторые камеры можно поворачивать через приложение для смартфона или ПК, а также увеличивать картинку. Есть камеры с функцией двусторонней связи, чтобы не только слышать, но и разговаривать с домашним питомцем. Перед отъездом нужно провести пару экспериментов с уровнем звука камеры и поставить её на таком расстоянии, чтобы животное могло слышать голос хозяина.

Многие камеры оснащены инфракрасной подсветкой, они могут снимать животных, которые ведут ночной образ жизни. Камеры с хорошим разрешением позволяют следить за мелкими животными — рептилиями или рыбками в аквариуме.

Камеру можно поставить на стол, подоконник или полку и примотать скотчем или придавить её основание тяжёлым предметом, чтобы животное не сдвинуло конструкцию. Провода нужно максимально спрятать: питомец может за них зацепиться.

Для наблюдения за рыбками и черепахами камеру можно поставить прямо напротив аквариума. Чтобы следить за активной кошкой, нужно поставить камеру повыше для обеспечения обзора всего помещения.

Можно использовать для наблюдения несколько камер. Одна может снимать комнату целиком, а другая — смотреть на место, где любит отдыхать кошка или собака. Камера с широким углом обзора также может использоваться в охранных целях.



\section{Ноутбук или смартфон}

Организовать видеонаблюдение можно с помощью смартфона, планшета или ноутбука. Ноутбук нужно поставить так, чтобы в зону видимости камеры попадало то, что вы хотите снимать. Для планшета или смартфона придётся докупить держатель. Лучше взять вариант на гибкой ножке, чтобы можно было настроить кадр.

Затем нужно установить программу для видеонаблюдения и удалённо подключиться через мобильное приложение или сайт. Нужно не забыть поставить устройство на зарядку: оно должно работать непрерывно.

Современные программы для видеонаблюдения, например Xeoma, умеют отправлять уведомления о событиях по СМС, почте или в мессенджерах. Так можно отследить, когда любимая кошка пришла на кухню поесть или спряталась под диван. Особенно удобно использовать эту функцию, когда в квартире установлено несколько камер. В Xeoma также есть детектор звука — хозяин может получить уведомление, когда питомец грустит и скулит.

\section{Профессиональное оборудование}

Комплект профессионального оборудования состоит из нескольких камер и рекордера, к которому они подключаются. Стоит такой набор в несколько раз дороже обычной IP-камеры, а его настройку лучше доверить профессионалу. Дополнительно нужно купить жёсткие диски для хранения видео и развести проводку по квартире.


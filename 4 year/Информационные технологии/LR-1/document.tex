% !TeX program = xelatex

%%% Загружаем заголовочный файл, который хранит все настройки и все
%%% подгружаемые пакеты
\newcommand{\No}{\textnumero}

\input{input/header.tex}

%%% Загружаем настройки пакета eskdx, там нужно заполнить информацию
%%% о документе - ФИО авторов, название документов и т.п.
%%% Название документа
\ESKDtitle{ Стеганографатор-2000 }
\ESKDdocName{ Техническое задание }

\ESKDauthor{ Смирнов М. Г. }
\ESKDchecker{ Швед Д. В. }
\ESKDnormContr{ Швед Д. В. }

%%% Для титульника
\ESKDtitleApprovedBy{ Преподаватель }{ Швед Д.В. }
\ESKDtitleAgreedBy{ Преподаватель }{ Швед Д.В.}
%\ESKDtitleAgreedBy{ Должность второго согласовавшего }{ Фам. второго согл. }
%\ESKDtitleAgreedBy{ Должность третьего согласовавшего }{ Фам. третьего согл. }
\ESKDtitleDesignedBy{ Студент }{ Смирнов М.Г.}
%\ESKDtitleDesignedBy{ Должность второго автора }{ Фам. второго автора }

\ESKDdepartment{ Ведомство }
\ESKDcompany{ Предприятие }
\ESKDclassCode{ Код по классификатору }
\ESKDsignature{ ИТМО.ЛР.100301.КОБ.ОЗ.Т.ПЗ }

\ESKDdate{ 2019/06/13 }


\begin{document}

%%% Делаем титульник
\maketitle

%%% Делаем содержание
\tableofcontents

/Users/max/Google Диск/Documents/IFMO/4 year/Информационные технологии/LR-1/parts/1-intro.tex

\section{Основание для разработки}

Основанием для проведения разработки является Договор №2810 от 28 октября 2019 года. Договор согласован с Директором ООО <<ИТМО>> Швед Дарьей Викторовной, именуемым в дальнейшем Заказчиком, и утвержден Генеральным директором ООО «Вотэтода» Смирновым Максимом Григорьевичем, именуемым в дальнейшем Исполнителем, 28 октября 2019.
\section{Назначение разработки}

Функциональным назначением программы является предоставление пользователю возможности стеганографического сокрытия информации.

Программа должна эксплуатироваться в подразделениях Заказчика, имеющих необходимость к сокрытию информацию стеганографическими методами.

Конечными пользователями программы должны являться сотрудники подразделений Заказчика, имеющие доступ к служебной информации.
\section{Технические требования к программе}

\subsection{Требования к функциональным характеристикам}

ПО "Стеганографатор-2000" должно выполнять следующие функции:


\subsubsection{Общесистемные функции}
\begin{itemize}
	\item Аутентификация и авторизация пользователей
	\item Создание запроса на восстановление пароля
	\item Вывод статистики используемых ресурсов каждым пользователем
	\item Поддержка алгоритма шифрования путем встраивания данных в пространственной области для файлов формата BMP и GIF
	\item Поддержка алгоритмов шифрования методом сокрытия в служебных областях для файлов формата WAV.
\end{itemize}

\subsubsection{Административные функции}
\begin{itemize}
	\item Создание пользователей ПО
	\item Управление правами пользователей ПО
	\item Проверка права пользователей на работу со стеганографическим контейнером
\end{itemize}

\subsubsection{Функции шифрования}
\begin{itemize}
\item Создание стеганографического контейнера в произвольном файле формата BMP, GIF или WAV.
\item Помещение информационного файла в подготовленный стеганографический контейнер
\end{itemize}

\subsubsection{Функции дешифрования}
\begin{itemize}
	\item Проверка наличия стеганографического контейнера в произвольном файле формата BMP, GIF или WAV.
	\item Извлечение информационного файла из стеганографического контейнера
	\item Удаление информационного файла из стеганографического контейнера
\end{itemize}




\subsection{Требования к надежности}

Надежное функционирование программы должно быть обеспечено выполнением Заказчиком совокупности организационно-технических мероприятий, перечень которых приведен ниже:

\begin{itemize}
\item организацией бесперебойного питания технических средств;
\item использованием лицензионного программного обеспечения;
\item регулярным выполнением рекомендаций Министерства труда и социального развития РФ, изложенных в Постановлении от 23 июля 1998г. <<Об утверждении межотраслевых типовых норм времени на работы по сервисному обслуживанию ПЭВМ и оргтехники и сопровождению программных средств>>;
\item соблюдением государственных требований и регламентов в области обеспечения информационной безопасности, а также внутренних корпоративных регламентов Заказчика по обеспечению информационной безопасности.

\end{itemize}

\subsection{Условия эксплуатации}

Температура воздуха в помещении должна находиться в пределах 20-24°C, влажность воздуха не должна превышать 70\%. Атмосферное давление должно быть на уровне 760 мм ртутного столба. Эти условия соответствуют условиям эксплуатации современных офисных компьютеров, а также обеспечивают комфортное самочувствие сотрудников.

Рабочие станции с установленной программой должны обслуживаться специалистом в области информационной безопасности. Требования к сотрудникам, имеющим доступ к рабочим станциям с установленной программой, требования к специалистам по информационной безопасности, осуществляющим сервисное обслуживание, а также интервалы сервисного обслуживания должны соответствовать государственным, а также внутрикорпоративным регламентам, установленным на предприятии Заказчика.

\subsection{Требования к составу и параметрам технических средств}

В состав технического средства должен входить IBM-совместимый персональный компьютер (ПЭВМ), включающий в себя:
\begin{itemize}
	\item двухъядерный процессор с тактовой частотой каждого ядра не менее 2 ГГц; 
	\item оперативную память объемом не менее 4Гб и частотой не менее 1333МГц;
	\item жесткий диск объемом не менее 160Гб;
\end{itemize}

\subsection{Требования к информационной и программной совместимости}

Программа должна быть совместима с операционной системой Windows 7.
Исходные коды программы должны быть реализованы на языке C++ стандарта ISO/IEC 14882:2017.
В качестве среды разработки должна быть использована среда Borland C++ версии 5.13.

\subsection{Требования к маркировке и упаковке}

Программа поставляется в виде программного изделия на внешнем оптическом носителе.

Упаковка изделия должна осуществляться в упаковочную тару предприятия-изготовителя.

\subsection{Требования к транспортированию и хранению}

При транспортировании и хранении программного изделия должна быть обеспечена защита от попадания пыли и атмосферных осадков.
Температура окружающего воздуха должна находиться в пределах 5-30°C, относительная влажность воздуха не должна превышать 70\%, атмосферное давление должно быть в районе 760 мм ртутного столба.
/Users/max/Google Диск/Documents/IFMO/4 year/Информационные технологии/LR-1/parts/6-indicators.tex
\section{Стадии и этапы разработки}

Разработка должна быть проведена в четыре стадии: 

\subsection{Разработка технического задания}
\begin{enumerate}
	\item Разработка технико-экономического обоснования разработки программы.
	\item Определение стадий, этапов и сроков разработки программы и документации на неё.
	\item Согласование и утверждение технического задания.
\end{enumerate}
\subsection{Разработка технического проекта}
\begin{enumerate}
	\item Определение формы представления входных и выходных данных.
	\item Разработка структуры программы.
	\item Разработка плана мероприятий по разработке и внедрению программ.
	\item Разработка пояснительной записки.
	\item Согласование и утверждение технического проекта.
\end{enumerate}

\subsection{Разработка рабочего проекта}
\begin{enumerate}
	\item Программирование и отладка программы.
	\item Разработка программной документации.
	\item Разработка, согласование и утверждение программы и методики испытаний.
	\item Проведение предварительных испытаний.
	\item Корректировка программы и программной документации по результатам испытаний.
\end{enumerate}
\subsection{Внедрение}
\begin{enumerate}
	\item Подготовка и передача программы.
	\item Оформление и утверждение акта о передаче программы.
\end{enumerate}
\section{Порядок контроля и приемки}

Приемо-сдаточные испытания должны проводиться на объекте Заказчика в срок до 20 июня 2020 года.

Приемо-сдаточные испытания должны проводиться согласно разработанной (не позднее 30 октября 2019 года) Исполнителем и согласованной с Заказчиком Программы испытаний.

Ход проведения приемо-сдаточных испытаний Исполнитель и заказчик документируют в Протоколе проведения испытаний.

%%% Далее выводим библиографию
%%% Исправляем ошибку библиографии - «кавычка перед тире»
%%% http://ru-tex.livejournal.com/105178.html?thread=801498
\catcode`"\active\def"{\relax}
\bibliographystyle{gost780s}
%\bibliography{bibliography}{}
%%% Если пропадают инициалы, смотрите сюда:
%%% http://plumbum-blog.blogspot.com/2010/10/bibtex-miktex-gost780s.html
\end{document}

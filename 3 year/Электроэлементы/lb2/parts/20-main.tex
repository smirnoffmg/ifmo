\section{Ход работы}


\subsection{Измерения}

\begin{table}[H]
	\centering
	\resizebox{\textwidth}{!}{%
		\begin{tabular}{|c|c|c|c|c|c|c|}
			\hline
			Резистор  & Номинал & 24     & 40     & 60     & 80     & 100    \\ \hline
			МЛТ       & 180 Ом  & 188 Ом & 188 Ом & 188 Ом & 188 Ом & 190 Ом \\ \hline
			С5-5 2Вт & 62 Ом   & 61 Ом  & 62 Ом  & 61 Ом  & 61 Ом  & 61 Ом  \\ \hline
			SCK-203    & 20 Ом   & 19 Ом  & 13 Ом  & 9 Ом   & 5 Ом   & 3 Ом   \\ \hline
			УЛМ       & 750 Ом  & 737 Ом & 735 Ом & 733 Ом & 731 Ом & 725 Ом \\ \hline
			ВС-2      & 200 Ом  & 202 Ом & 201 Ом & 200 Ом & 199 Ом & 198 Ом \\ \hline
		\end{tabular}%
	}
\end{table}

\subsection{МЛТ}

Резисторы МЛТ, ОМЛТ постоянные, непроволочные, неизолированные, теплостойкие.

Предназначены для работы в цепях постоянного, переменного и импульсного тока в качестве элементов
навесного монтажа.

\begin{figure}[H]
	\centering
	\begin{tikzpicture}
	\begin{axis}[
	title={Зависимость сопротивления от температуры для МЛТ},
	xlabel={$t$, $^\circ C$},
	ylabel={$R$, Ом },
	xmin=0, xmax=110,
	ymin=170, ymax=200,
	legend pos=north west,
	ymajorgrids=true,
	grid style=dashed,
	]
	
	\addplot[
	color=blue,
	mark=square,
	smooth
	]
	coordinates {
		(24, 188) (40, 188) (60, 188) (80, 188) (100, 190)
	};
	
	
	\end{axis}
	\end{tikzpicture}
\end{figure}

\[
\alpha_{c_\text{МЛТ}} = \frac{190-188}{ 190 \cdot (100-40)} \approx 0.0001
\]


\subsection{С5-5 2Вт}
Резисторы С5-5 постоянные проволочные.
Прецизионные и общего применения, изолированные, для навесного монтажа. 
Предназначены для работы в цепях постоянного и переменного токов с частотой до 1 кГц. 

\begin{figure}[H]
	\centering
	\begin{tikzpicture}
	\begin{axis}[
	title={Зависимость сопротивления от температуры для С5-5 2Вт},
	xlabel={$t$, $^\circ C$},
	ylabel={$R$, Ом },
	xmin=0, xmax=110,
	ymin=60, ymax=63,
	legend pos=north west,
	ymajorgrids=true,
	grid style=dashed,
	]
	
	\addplot[
	color=blue,
	mark=square,
	smooth
	]
	coordinates {
		(24, 62) (40, 61) (60, 62) (80, 61) (100, 61)
	};
	\end{axis}
	\end{tikzpicture}
\end{figure}

\[
\alpha_{c_\text{С5-5 2Вт}} = \frac{62-61}{ 62 \cdot (60-40)} \approx 0.0008
\]

\subsection{SCK-203}

\begin{figure}[H]
	\centering
	\begin{tikzpicture}
	\begin{axis}[
	title={Зависимость сопротивления от температуры для SCK-203},
	xlabel={$t$, $^\circ C$},
	ylabel={$R$, Ом },
	xmin=0, xmax=110,
	ymin=0, ymax=21,
	legend pos=north west,
	ymajorgrids=true,
	grid style=dashed,
	]
	
	\addplot[
	color=blue,
	mark=square,
	smooth
	]
	coordinates {
		(24, 19) (40, 13) (60, 9) (80, 5) (100, 3)
	};
	\end{axis}
	\end{tikzpicture}
\end{figure}

\[
\alpha_{c_\text{SCK-203}} = \frac{3-13}{ 3 \cdot (100-40)} \approx -0.06
\]

\subsection{УЛМ}

С таким названием с конца 50-х по начало 70-х годов выпускались резисторы \textbf{ВС-0.125} (надо отметить, что при этом одновременно были и резисторы с названием \textbf{ВС}).

Они предназначены для использования в малогабаритной аппаратуре, например, с полупроводниковыми приборами

\begin{figure}[H]
	\centering
	\begin{tikzpicture}
	\begin{axis}[
	title={Зависимость сопротивления от температуры для УЛМ},
	xlabel={$t$, $^\circ C$},
	ylabel={$R$, Ом },
	xmin=0, xmax=110,
	ymin=720, ymax=740,
	legend pos=north west,
	ymajorgrids=true,
	grid style=dashed,
	]
	
	\addplot[
	color=blue,
	mark=square,
	smooth
	]
	coordinates {
		(24, 737) (40, 735) (60, 733) (80, 731) (100, 725)
	};
	\end{axis}
	\end{tikzpicture}
\end{figure}

\[
\alpha_{c_\text{УЛМ}} = \frac{725-735}{ 725 \cdot (100-40)} \approx -0.0002
\]

\subsection{ВС-2}

Резисторы постоянные углеродистые ВС предназначены для работы в цепях постоянного, переменного и импульсного тока в радиотехнической и электронной аппаратуре.

Резисторы предназначены также для работы в условиях сухого и влажного тропического климата.

\begin{figure}[H]
	\centering
	\begin{tikzpicture}
	\begin{axis}[
	title={Зависимость сопротивления от температуры для ВС-2},
	xlabel={$t$, $^\circ C$},
	ylabel={$R$, Ом },
	xmin=0, xmax=110,
	ymin=197, ymax=203,
	legend pos=north west,
	ymajorgrids=true,
	grid style=dashed,
	]
	
	\addplot[
	color=blue,
	mark=square,
	smooth
	]
	coordinates {
		(24, 202) (40, 201) (60, 200) (80, 199) (100, 198)
	};
	\end{axis}
	\end{tikzpicture}
\end{figure}

\[
\alpha_{c_\text{ВС-2}} = \frac{198-201}{ 198 \cdot (100-40)} \approx -0.0003
\]



\section{Вывод}

Было проведено исследование зависимости сопротивления резисторов от температуры и определение ТКС различных типов резисторов.

Графики, полученные в ходе работы, подтверждают теоретические сведения.


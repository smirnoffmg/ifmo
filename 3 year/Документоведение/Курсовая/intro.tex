\section*{Введение}
\addcontentsline{toc}{section}{Введение}%

Документооборот — движение документов в организации с момента их создания или получения до завершения исполнения или отправления\cite{GOST7082013}; комплекс работ с документами: приём, регистрация, рассылка, контроль исполнения, формирование дел, хранение и повторное использование документации, справочная работа.

Объём документооборота компании постоянно возрастает и даже в небольших учреждениях измеряется тысячами, а в развитых и крупных компаниях сотнями тысяч документов ежегодно. При такой активности остро начинает вставать вопрос об автоматизации данного процесса, а именно контроль и способы хранения всей обрабатываемой информации. Для решения таких задач существует класс систем СЭД - системы электронного документооборота.

С развитием компьютерных технологий начало изменяться и техническое оснащение аппарата управления, что привело к рождению нового подхода к обеспечению деятельности т.н. общего отдела – <<электронный офис>>.

Электронным офисом называется программно-аппаратный комплекс, предназначенный для обработки документов и автоматизации работы пользователей в системах управления.

\textbf{Целью} данной работы является рассмотрение понятия <<электронный офис>> в практике современных коммерческих организаций и учреждений.

Будем рассматривать жизненный цикл предприятия как рост от микробизнеса до крупного бизнеса, \textbf{задачей} работы будет рассмотрение процессов обеспечения функционирования предприятия на каждом этапе и то, как могут ли эти процессы быть перенесёнными в электронную форму.

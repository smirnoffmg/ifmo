\documentclass[a4paper,12pt,russian]{extreport}
 
\usepackage{extsizes}
\usepackage{cmap} % для кодировки шрифтов в pdf
\usepackage[T2A]{fontenc}
\usepackage[utf8]{inputenc}
\usepackage[russian]{babel}
  
\usepackage{amssymb,amsfonts,amsmath,amsthm} % математические дополнения от АМС
\usepackage{indentfirst} % отделять первую строку раздела абзацным отступом тоже
\usepackage[usenames,dvipsnames]{color} % названия цветов
 
\renewcommand{\rmdefault}{ftm} % Times New Roman
\frenchspacing

\usepackage{geometry}
\geometry{left=1.5cm}
\geometry{right=1.5cm}
\geometry{top=2.4cm}
\geometry{bottom=2.4cm}


\begin{document}
\begin{enumerate}
\item Комплексные числа: определение, действия над ними. Сопряженное комплексное число. Алгебраическая, тригонометрическая и показательная формы комплексного числа. Формула Эйлера. Формула Муавра. Корень n-ой степени из 1.
\item Определители 2-го и 3-го порядков: определение, основные свойства и способы их вычисления. Миноры и алгебраические дополнения для определителей 2-го и 3-го порядков. Правило Крамера для решения систем линейных уравнений с двумя и тремя неизвестными.
\item Определители 4-го порядка и выше: определение, основные свойства и способы их вычисления. Миноры и алгебраические дополнения для определителей n-го порядка. Правило Крамера для решения систем линейных уравнений с n неизвестными.
\item Матрицы: определение, основные виды матриц. Сложение, вычитание, умножение матрицы на число, умножение матриц друг на друга. Обратная матрица. Матричный метод решения систем линейных уравнений.
\item Линейная зависимость и независимость столбцов и строк матрицы. Минор k-того порядка. Определение ранга матрицы.
\item Совместность системы с m линейными уравнений с n неизвестными. Определение решения системы, совместная и несовместная система, определённая и неопределённая системы. Понятие расширенной матрицы системы. Теорема Кронекера-Капелли. Элементарные преобразования системы. Метод Гаусса (метод исключения неизвестных) для решения системы m линейных уравнений с n неизвестными.
\item Система прямоугольных декартовых координат на плоскости. Расстояние между точками на плоскости. Деление отрезка в данном отношении. Уравнение линии в прямоугольных декартовых координатах на плоскости. Центр тяжести трегольника. Уравнение медиан, биссектрис и высот в треугольнике.
\item Прямая линия на плоскости.
  \begin{enumerate}
  \item Уравнение прямой с угловым коэффицициентом, общее уравнение прямой, уравнение прямой в отрезках.
  \item Угол между двумя прямыми. Условие параллельности и перпендикулярности прямых. Пересечение двух прямых.
  \item Уравнение прямой, проходящей через данную точку в данном направлении. Пучок прямых, уравнение прямой, проходящей через две данные точки.
  \item Нормальное уравнение прямой. Расстояние от точки до прямой. Расположение точек относительно прямой: по одну сторону от прямой и по разные стороны от прямой.
  \end{enumerate}
\item Линии второго порядка на плоскости.
  \begin{enumerate}
  \item Окружность: приведение к каноническому виду общего уравнение окружности.
  \item Эллипс: определение, каноническое уравнение эллипса, полуоси, фокусы, фокальные радиусы, эксцентриситет, директрисы эллипса. Свойства директрисы эллипса.
  \item Гипербола: определение, каноническое уравнение, полуоси, характеристический прямоугольник, фокусы, фокальные радиусы, асимпоты, директрисы и эксцентриситет гиперболы.
  \item Парабола: определение, каноническое уравнение параболы относительно осей OX и OY, директриса, фокус, фокальный радиус параболы.
  \end{enumerate}
\item Преобразование прямоугольных декартовых координат на плоскости: параллельный перенос (сдвиг) осей и поворот осей. Упрощение уравнений кривых второго порядка. Приведение уравнений кривых второго порядка к каноническому виду с помощью преобразований прямоугольных декартовых координат: поворота и сдвига.
\item Задание плоских кривых в полярных координатах и параметрически.
\item Декартовы прямоугольные координаты в пространстве: полупространства и октанты, координатные плоскости и оси. Расстояние между точками. Деление отрезка в данном отношении.
\item Понятие вектора. Виды векторов: равные, единичные, взаимно обратные, коллинеарные, компланарные. Проекция вектора на ось. Координаты и направляющие косинусы вектора. Линейные операции над векторами: сложение и умножение на число и их свойства. Признак коллинеарности двух векторов. Координатный базис.
\item Скалярное, векторное и смешанное произведение векторов: определение, свойства, геометрический смысл, их нахождение в векторной и координатной форме. Двойное векторное произведение.
\item Уравнение поверхности и линии в пространстве, уравнениея цилиндрической поверхности с образующими, параллельными одной из координатных осей. Уравнение плоскости, проходящей через заданную точку и имеющей данный нормальный вектор с выводом, общее уравнение плоскости.
\item Неполные уравнений плоскостей. Уравнение плоскости в отрезках. Нормальное уравнение плоскости. Расстояние от точки до плоскости. Угол между двумя плоскостями. Условие параллельности и перпендикулярности двух плоскостей. Пучок плоскостей.
\item Уравнение прямой в пространстве, направляющий вектор прямой. Каноническое и параметрическое уравнение прямой с выводом. Прямая как пересечение двух плоскостей в пространстве. Связь между различными видами уравнений прямой в пространстве.
\item Взаимное расположение двух прямых в пространстве: параллельность прямых, скрещивание прямых, принадлежность двух прямых одной плоскости. Угол между двумя прямыми. Расстояние от точки до прямой. Кратчайшее расстояние между двумя прямыми. Прямая и плоскость в пространстве.
\item Виды повехностей второго порядка в пространстве: сфера, цилиндрические и канонические поверхности, эллипсоид, однополостный и двухполостный гиперболоиды, параболоиды: их канонические уравнения и основные свойства. Приведение общего уравнения поверхности второго порядка к каноническому виду. Формулы преобразования прямоугольных декартовых координат в пространстве:
  \begin{enumerate}
  \item при параллельном переносе координатных осей
  \item при повороте координатых осей
  \end{enumerate}
\item Собственные числа и собственные вектора матриц. Использование понятий собственных чисел и собственных векторов для приведения к каноническому виду:
  \begin{enumerate}
  \item уравнений кривых плоских 2-го порядка
  \item уравнений поверхности 2-го порядка
  \end{enumerate}
\end{enumerate}
\end{document}

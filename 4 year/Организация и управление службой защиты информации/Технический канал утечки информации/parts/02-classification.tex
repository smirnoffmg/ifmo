\section{Классификация ТКУ}

Существует несколько классификаций технических каналов утечки информации. Например, по причине возникновения различают:

\begin{itemize}
	\item Естественные ТКУ (к ним относятся ТКУ, образующиеся за счет несовершенства конструкторско-технологических качеств технических средств хранения, обработки и передачи информации);
	\item искусственно созданные ТКУ (организуются преднамеренно, как правило, за счет внешних энергетических воздействий на носитель или коммуникации).
\end{itemize}

Классификацию по виду информации нагляднее представить в виде схемы.

\begin{figure}[H]
	\centering
	\includegraphics[width=0.7\linewidth]{"img/Tcoibl6_2"}
	\caption{Классификация ТКУ по виду информации}
\end{figure}

Надо понимать, что на вход канала поступает информация в виде первичного сигнала. Первичный сигнал представляет собой носитель с информацией от ее источника или с выхода предыдущего канала. В качестве источника сигнала могут быть:

\begin{itemize}
\item объект наблюдения, отражающий электромагнитные и акустические волны;
\item объект наблюдения, излучающий собственные (тепловые) электромагнитные волны в оптическом и радиодиапазонах;
\item передатчик функционального канала связи;
\item закладное устройство;
\item источник опасного сигнала;
\item источник акустических волн, модулированных информацией.
\end{itemize}

Так как информация от источника поступает на вход канала на языке источника (в виде буквенно-цифрового текста, символов, знаков, звуков, сигналов и т. д.), то передатчик производит преобразование этой формы представления информации в форму, обеспечивающую запись ее на носитель информации, соответствующий среде распространения. В общем случае он выполняет следующие функции:

\begin{itemize}
	\item создает поля или электрический ток, которые переносят информацию;
	\item производит запись информации на носитель;
	\item 	усиливает мощность сигнала (носителя с информацией);
	\item 	обеспечивает передачу сигнала в среду распространения в заданном секторе пространства.
\end{itemize}

Среда распространения носителя - часть пространства, в которой перемещается носитель. Она характеризуется набором физических параметров, определяющих условия перемещения носителя с информацией. Основными параметрами, которые надо учитывать при описании среды распространения, являются:

\begin{itemize}
	\item физические препятствия для субъектов и материальных тел:
	\item мера ослабления сигнала на единицу длины;
	\item 	частотная характеристика;
	\item 	вид и мощность помех для сигнала.
\end{itemize}

Приемник выполняет функции, обратные функциям передатчика. Он производит:

\begin{itemize}
\item выбор носителя с нужной получателю информацией;
\item усиление принятого сигнала до значений, обеспечивающих съем информации;
\item съем информации с носителя;
\item преобразование информации в форму сигнала, доступную получателю (человеку, техническому устройству), и усиление сигналов до значений, необходимых для безошибочного их восприятия.
\end{itemize}

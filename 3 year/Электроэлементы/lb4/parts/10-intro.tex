\section{Введение}

\subsection{Цель работы}

Цель работы - ознакомление с основными параметрами и характеристиками оптрона типа светоизлучающий диод - фототиристор.

\subsection{Краткие теоретические сведения}

Оптроном называется оптоэлектронный прибор, в котором имеются
управляемый источник излучения и фотоприемник с тем или иным видом оптической связи между ними, конструктивно объединенные и помещенные в один корпус.

В оптроне энергия электрического сигнала с помощью светоизлучающего диода преобразуется в световую, затем через оптическую среду передается на фотоприемник, в котором снова происходит преобразование энергии света в электрический сигнал. Такое преобразование энергии позволяет передавать информацию из одной электрической цепи в другую с помощью электрически нейтральных фотонов. Это свойство определяет ряд преимуществ оптрона:
\begin{itemize}
	\item  практически полную гальваническую развязку входа и выхода (достижимо сопротивление изоляции $10^{12}$ - $10^{14}$ Ом, емкость связи не более $10^{-14}$ Ф),
	\item  однонаправленность потока информации и обусловленное этим отсутствие обратной реакции приемника на источник, возможность создания сильно разветвленных коммутаций, нагруженных на "несогласованные" разнородные потребители энергии,
	\item  возможность реализации бесконтактных (механически и электрически) связей, например через воздушную среду,
	\item  физическое и конструктивное разнообразие, широта функциональных возможностей и др
\end{itemize}

Оптроны применяются в качестве развязывающих (изолирующих) четырехполюсников в цепях постоянного и переменного токов, в импульсных и высоковольтных цепях. С их помощью легко согласовать между собой низкоомные и высокоомные, высоковольтные и низковольтные, высокочастотные и низкочастотные цепи. Они применяются в качестве реле для коммутации напряжений и токов, аналоговых преобразователей, оптических разъемов и пр.

Наряду с этим оптроны характеризуются такими достоинствами, как высокое быстродействие, малые габариты, отсутствие механических подвижных контактов, широкий частотный диапазон, большой срок службы и надежность.

Наиболее распространенными оптоэлектронными парами, применяемыми в оптронах, являются GaAs - светоизлучающие диоды и фотоприемники на основе кремния, которые хорошо согласуются между собой по спектру излучения и поглощения, а также по быстродействию.

Согласование спектральных характеристик источника излучения и фотоприемника является одним из основных условий, обеспечивающих оптимальную передачу сигнала с входа оптрона на его выход. В качестве передающей среды используются газовый промежуток (в том числе воздушный), различные иммерсионные, согласующие среды (полимерные органические оптические лаки и клеи, низкотемпературные халькогенидные стекла) и оптоволоконные линии (световоды) - жесткие стержни, гибкие жгуты. Оптическая среда выбирается таким образом, чтобы она обеспечивала высокую пробивную прочность.

Большинство типов промышленных оптронов рассчитывается на работу совместно с интегральными микросхемами. Это требует согласования их входных параметров с ДТЛ, ТТЛ и другими интегральными микросхемами. Если источник света в оптроне является светоизлучающий диод, то его входные параметры определяются электрическими параметрами СИД: $U_\text{ВХ}$ = 1,25 - 1,6 В; $I_\text{ВХ}$ = 20 - 100 мА; $U_\text{ОБР}$ ≤3,5В;$I_\text{ОБР}$ =10мкА;$C_\text{Д}$ =100-150пФ.

Выходные параметры определяются выбранным типом фотоприемника. Распространение получили оптроны на фотодиодах и фототранзисторах для работы с микросхемами, а также на фоторезисторах и фототиристорах для целей автоматики и измерительной техники.

Наименьший коэффициент передачи тока - у диодных оптронов, для которых K=1...1,5\%, однако диодные оптроны являются самыми быстродействующими вплоть до $t_\text{ВКЛ} \approx 10^{-8}$ с.

Транзисторные оптроны характеризуются наибольшей схемотехнической гибкостью, имеют высокое значение коэффициента
передачи тока K = 50 - 100\%, но относительно невысокое быстродействие $t_\text{ВКЛ} (t_\text{ВЫКЛ}) \approx 2 - 5 мкс$.

Специфическими для тиристорных оптопар статическими параметрами являются:
\begin{itemize}
\item входной ток срабатывания $I_\text{ВХ.СРАБ.}$ - постоянный прямой входной ток, который переводит оптопару в открытое состояние при заданном режиме на выходе,
\item выходной ток в закрытом состоянии $I_\text{ВЫХ.ЗАКР.}$ - ток протекающий в выходной цепи при закрытом состоянии фототиристора и заданном режиме,
\item выходное напряжение в открытом состоянии $U_\text{ВЫХ.ОТКР.}$ - напряжение на выходных выводах тиристорной оптопары в условиях открытого состояния фототиристора,
\item выходной удерживающий ток $I_\text{ВЫХ.УД.}$ - наименьшее значение выходного тока, при котором фототиристор еще находится в открытом состоянии в отсутствии входного тока,
\item выходной минимальный ток при подаче управляющего сигнала $I_\text{ВЫХ.МИН}$. 
\item минимальное значение выходного тока, при котором фототиристор сохраняет включенное состояние при наличии входного сигнала.
\end{itemize}

Тиристорные оптроны используются как мощные ключи, имеющие хорошую электрическую изоляцию между цепью управления и анодной цепью. Управляя значительными мощностями в нагрузке (до 100 кВт), тиристорные оптопары по входу практически совместимы с интегральными микросхемами.

Фототиристор, так же как и обычный тиристор, имеет четырехслойную структуру $p-n-p-n$. Конструктивно оптопара выполнена так, что основная часть излучения входного СИД направлена на высокоомную базовую область $n$ фототиристора.

К крайним областям - аноду и катоду - прикладывается внешнее выходное напряжение плюсом к аноду.

В случае затемненной n базы под действием внешнего напряжения через тиристор протекает незначительный обратный ток, обусловленный
прохождением неосновных носителей через обратносмещенный центральный $p-n$ переход (тиристор закрыт). Проходящие через переход неосновные носители накапливаются в $n$ и $p$ базах, увеличивая их потенциал. Увеличение отрицательного заряда в $n$ базе и положительного заряда в $p$ базе приводит к увеличению инжекции носителей (дырок $p$ анодом и электронов $n$ катодом) в базовые области. Инжектированные носители беспрепятственно проходят центральный переход и накапливаются в базовых областях, которые являются потенциальными ямами для электронов ($n$ база) и дырок ($p$ база). При определенном значении напряжения, называемом напряжением "опрокидывания", этот процесс носит лавинообразный характер, все три $p-n$ перехода оказываются открытыми, ток через прибор резко возрастает, а напряжение на структуре падает (тиристор открыт).

При освещении $n$ базы под действием светового потока в ней генерируются пары электронов и дырок. Под действием электрического поля центрального $p-n$ перехода происходит разделение генерированных носителей, в результате чего заряды $n$ и $p$ баз возрастают. Накопление дополнительных зарядов в центральных областях приводит к тому, что <<опрокидывание>> тиристора происходит при меньшем напряжении.

Фототиристор обладает большим внутренним усилением тока. В отличие от других типов оптопар открытое состояние фототиристора сохраняется и при прекращении излучения входного диода. Чтобы закрыть фототиристор, с него надо снять внешнее напряжение.
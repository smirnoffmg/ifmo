\chapter{Основная часть}

\section{Определение почтовых адресов}

С использованием Google найдём сайт организации (стр. \pageref{asty}).

Сайт ASTY.PRO расположен по адресу https://asty.pro/.

Проанализируем правила формирования почтовых учетных записей в организации из раздела <<Контакты>> (стр. \pageref{contacts}).


К сожалению, ничего интересного мы там найти не сможем. На сайте и в социальных сетях указан единственный email для обращений - info@asty.pro.

\section{Поиск документов с расширением PDF на домене организации}

Документы не представляют собой ничего интересного, однако стали видны поддомены организации (стр. \pageref{docs}).


\section{Поиск документов с расширением PDF, содержащие упоминание организации}

Единственное ценное, что удалось найти - их разработчик Андрей Жгилёв ищет работу. Можно было бы позвонить и представиться рекрутёром, но это выходит за рамки данной лабораторной работы (стр. \pageref{mime_pdf}).

\section{Изучение кэшированных копий сайта организации}

Ничего интересного, сайт почти не менялся.

\section{Обнаружение скрытых от поисковиков URLs в файле robots.txt на домене выбранной организации}


По результатам анализа <<robots.txt>> мы определяем на чем написан сайт (MODX) и адрес панели администратора (https://asty.pro/manager/) (стр. \pageref{modx}).

\section{Определение диапазон сети, в который входит IP-адрес сайта, и информацию о домене с помощью утилиты whois}

Владелец доменного имени - Михаил Хандобин, проживающий по адресу гор. Санкт-Петербург, Загребский бульвар, дом 9, квартира 1129, телефон +7 (921) 903-11-48 (стр. \pageref{whois}).

\section{Обнаружение субдоменов основного домена организации}

Тут тоже ничего интересного. Поддомены используются для работы над заказами (стр. \pageref{sublstr})


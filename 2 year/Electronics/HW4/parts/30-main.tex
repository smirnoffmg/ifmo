\chapter{Расчёт схемы}

Коэффициент усиления каскада:
\[
K_{NI}=1+\frac{R2}{R1}=2
\Rightarrow \frac{R2}{R1} = 1
\]

Коэффициент обратной связи:

\[
K_{FB}=\frac{R1}{R1+R2}=0.5
\]

Выбран операционный усилитель с низким уровнем шума \textbf{OPA177}. Параметры представлены в разделе <<Выбор компонентов>>.

Коэффициент усиления без обратной связи для частоты в 200 Гц составляет 95 Дб (по рисунку 3.2 в разделе <<Выбор компонентов>>).
Для перевода из Дб в разы используем следующую формулу:

\[
K_{OL}=10^{\frac{95}{10}}=3.16 \cdot 10^7
\]

Среднеквадратическое напряжение шума, мкВ:
\[
U_{NOISE}=0.09 
\]

Среднеквадратическое значение шумового тока, пА:
\[
I_{NOISE}=0.4 \cdot \sqrt{200}=5.66 
\]

Пусть cопротивление $R1$=10 кОм, тогда исходя из соотношения выше получим, что сопротивление $R2$=10 кОм.

Выбраны резисторы \textbf{CF-100} (С1-4) номиналом 10 кОм, характеристики представлены в разделе <<выбор компонентов>>.

Сопротивление $R3$, кОм:

\[
R3=\frac{R1 \cdot R2}{R1 + R2}=5
\]

Такое сопротивление можно получить параллельным соединением двух резисторов в 10 кОм, которые были выбраны ранее.

Среднеквадратическое значение напряжения шума:

\begin{multline*}
	U_{NOISE\_OUT}=\sqrt{\Delta f} \cdot \sqrt{\left(U_{NOISE}^2+4kTR3+4kTR1+I_{NOISE}^2\cdot R3  \right)\cdot \\ \left( 1 +	 \frac{R2}{R1} \right)^2 + I_{NOISE}^2\cdot R2^2 + 4kTR2} = 2.8 \cdot 10^{-6}
\end{multline*}

Амплитуда выходного напряжения, В:
\[
\Delta U_{OUT}=\Delta U_{IN} \cdot 2= 8
\]

Вычислим общую точность по формуле:
\[
E = E_K + \frac{U_{NOISE\_OUT}}{\Delta U_{OUT}}=4.1 \cdot 10^{-7}
\]
